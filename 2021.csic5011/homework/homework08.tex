\documentclass[11pt]{article}

\usepackage{amssymb}
\usepackage{amsmath}
\usepackage{graphicx}
\usepackage{hyperref}

\def\N{{\mathbb N}}
\def\NN{{\mathcal N}}
\def\R{{\mathbb R}}
\def\E{{\mathbb E}}
\def\rank{{\mathrm{rank}}}
\def\tr{{\mathrm{trace}}}
\def\P{{\mathrm{Prob}}}
\def\sign{{\mathrm{sign}}}
\def\diag{{\mathrm{diag}}}

\setlength{\oddsidemargin}{0.25 in}
\setlength{\evensidemargin}{-0.25 in}
\setlength{\topmargin}{-0.6 in}
\setlength{\textwidth}{6.5 in}
\setlength{\textheight}{8.5 in}
\setlength{\headsep}{0.75 in}
\setlength{\parindent}{0.25 in}
\setlength{\parskip}{0.1 in}

\newcommand{\lecture}[4]{
   \pagestyle{myheadings}
   \thispagestyle{plain}
   \newpage
   \setcounter{page}{1}
   \setcounter{section}{0}
   \noindent
   \begin{center}
   \framebox{
      \vbox{\vspace{2mm}
    \hbox to 6.28in { {\bf A Mathematical Introduction to Data Science \hfill #4} }
       \vspace{6mm}
       \hbox to 6.28in { {\Large \hfill #1  \hfill} }
       \vspace{6mm}
       \hbox to 6.28in { {\it Instructor: #2\hfill #3} }
      \vspace{2mm}}
   }
   \end{center}
   \markboth{#1}{#1}
   \vspace*{4mm}
}


\begin{document}

\lecture{Homework 8. Persistent Homology}{Yuan Yao}{Due: 1.5 weeks later}{Apr. 23, 2021}

The problem below marked by $^*$ is optional with bonus credits. % For the experimental problem, include the source codes which are runnable under standard settings. 

\begin{enumerate}
\item {\em Persistent Homology:} Download the Javaplex (latest version 4.3.4) from the following site

\url{http://appliedtopology.github.io/javaplex/}

Follow the tutorial at 

\url{https://github.com/appliedtopology/javaplex/wiki/Tutorial} 

to get it work under Matlab environment. For example, extract the zip file and open Matlab, change Matlab's ``Current Folder" to the directory where the \texttt{load\_javaplex.m} file is located (\texttt{src/matlab/for\_distribution/} in version 4.3.4), run the file: 

\begin{quote} \begin{verbatim} 
>> load_javaplex
\end{verbatim} \end{quote}

Also in the Matlab command window, type the following command.

\begin{quote} \begin{verbatim} 
>> import edu.stanford.math.plex4.*;
\end{verbatim} \end{quote}

Installation is complete. Confirm that Javaplex is working properly with the following command.

\begin{quote} \begin{verbatim}
>> api.Plex4.createExplicitSimplexStream()
ans = edu.stanford.math.plex4.streams.impl.ExplicitSimplexStream@16966ef
\end{verbatim} \end{quote}

Your output should be the same except for the last several characters. Each time upon starting a new Matlab session, you will need to run \texttt{load\_javaplex.m}.


Now conduct the following numerical experiment with the example shown in class: 
\begin{figure}[!h]
\centering
\includegraphics[width=0.4\textwidth]{persistence.jpg}  
\includegraphics[width=0.33\textwidth]{PersistentBettiNumbers.png}  \\
(a) \ \ \ \ \ \ \ \ \ \ \ \ \ \ \ \ \ \ \ \ \ \ \ \ \ \ \ \ \ \ \ \ \ \ \ \ \ \ \ \ \ \ \ \ \ \ \ \ \ (b)
\caption{(a) A filtration of simplices; (b) barcodes of Persistent Betti Numbers. \label{fig:filtration} }
\end{figure}


\begin{enumerate}
\item Construct a filtration (stream) of the simplicial complex in the Figure \ref{fig:filtration} (a). This could be done by 
\subitem 
\begin{quote} \begin{verbatim}
>> stream = api.Plex4.createExplicitSimplexStream(); 
>> stream.addVertex(0,0);
>> stream.addVertex(1,1);
>> stream.addVertex(2,2);
>> stream.addVertex(3,3);
>> stream.addElement([0,2],4);
>> stream.addElement([0,1],5);
>> stream.addElement([2,3],6);
>> stream.addElement([1,3],7);
>> stream.addElement([1,2],8);
>> stream.addElement([1,2,3],9);
>> stream.addElement([0,1,2],10);
>> stream.finalizeStream();
\end{verbatim} \end{quote}
\subitem where you can check the number of simplices in the filtration (stream) is 11
\begin{quote} \begin{verbatim} 
>> num_simplices = stream.getSize()
\end{verbatim} \end{quote}
\item Compute the persistent homology for the filtration and plot the barcode as Figure \ref{fig:filtration} (b).
\begin{quote} \begin{verbatim} 
>> % Compute the Z/2Z persistence homology of dimension less than 3:
>> persistence = api.Plex4.getModularSimplicialAlgorithm(3, 2); 
>> intervals = persistence.computeIntervals(stream);
>> options.filename = 'Persistent-Betti-Numbers';
>> options.max_filtration_value = 11;
>> % Plot the barcode of persistent Betti numbers:
>> plot_barcodes(intervals, options); 
\end{verbatim} \end{quote}
\end{enumerate}

\item {\em Torus example:} In \texttt{javaplex}, construct a torus, the Vietoris-Rips complex, persistent homology, and plot the barcodes of persistent Betti numbers. 

\begin{enumerate}
\item Load a pre-stored torus embeded in 4-dimensional space with 400 samples
\begin{quote} \begin{verbatim}
>> load tutorial_examples/pointsTorusGrid.mat
>> scatter3(pointsTorusGrid(:,1), pointsTorusGrid(:,2), pointsTorusGrid(:,3))
\end{verbatim} \end{quote}
\item Construct a Vietoris-Rips complex
\begin{quote} \begin{verbatim} 
>> max_dimension = 3;
>> max_filtration_value = 0.9;
>> num_divisions = 1000;
>> stream = api.Plex4.createVietorisRipsStream(pointsTorusGrid, ... 
    max_dimension, max_filtration_value, num_divisions);
>> num_simplices = stream.getSize()
\end{verbatim} \end{quote}
\item Compute the persistent homology for the Vietoris-Rips complex
\begin{quote} \begin{verbatim} 
>> persistence = api.Plex4.getModularSimplicialAlgorithm(max_dimension, 2);
>> intervals = persistence.computeIntervals(stream);
\end{verbatim} \end{quote}
\item Plot the barcode for the persistent Betti numbers, like Fig. \ref{fig:torus}
\begin{quote} \begin{verbatim} 
>> options.filename = 'ripsTorus4.png';
>> options.max_filtration_value = max_filtration_value;
>> options.max_dimension = max_dimension - 1;
>> options.side_by_side = true;
>> plot_barcodes(intervals, options);
\end{verbatim} \end{quote}
\item Explain the picture you find.
\end{enumerate}

\begin{figure}[!h]
\centering
\includegraphics[width=0.4\textwidth]{ripsTorus4.png}  
\caption{Barcodes of Persistent Betti Numbers of a Torus. \label{fig:torus} }
\end{figure}


\item {\em Single Linkage Clustering and Persistent 0-Homology}: Hierarchical clustering algorithms performs the following algorithm to build up a cluster tree.

\begin{enumerate}
\item Start with a set of $n$ data points as $n$ initial clusters (leaves). 
\item Repeatedly merge two ``closest" clusters.
\end{enumerate}

There are a variety of ``distance" (or ``dissimilarity") notions between clusters:
\begin{enumerate}
\item[(1)] {\bf Single Linkage}: distances (dissimilarity) between two \emph{closest} pair of points in two clusters;
\item[(2)] {\bf Complete Linkage}: distances (dissimilarity) between two \emph{furthest} pair of points in two clusters;
\item[(3)] {\bf Complete Linkage}: average pairwise distances (dissimilarity) between two clusters.
\end{enumerate}

     	\begin{figure}
                   		\centering
                   		\includegraphics[width=.6\linewidth]{../ISLR/graphics/ISLRFigures/10_12.pdf}
                   		%\caption{A subfigure}
                   		\caption{\scriptsize 10.12 (Introduction to Statistical Learning with Applications in R). Average, complete, and single linkage applied to an example
                   			data set. Average and complete linkage tend to yield more balanced clusters.
                   		}
        \end{figure}
        
Show that 
\begin{enumerate}
\item \emph{$0$-dimensional Persistent Homology of $n$ points in a metric space is equivalent to single linkage clustering without labeling, i.e. permutation invariant}; 
\item \emph{For a continuous density function $p:\R^d\to \R_+$, let 
$$C_p(\lambda):=\{\mbox{connected components of $x\in\R^d$}: p(x)\geq \lambda\} $$ 
be the connected components of super-level sets of $p(x)\geq \lambda\geq 0$, then $C_p(\lambda)\subseteq C_p(\lambda^\prime)$ as $\lambda \geq \lambda^\prime$ and $C_p(\lambda)$ induces a tree such that the vertex set at level $\lambda$ consists of connected components and an edge connects connected components at adjacent levels if $\sigma \subseteq\tau$ for $\sigma\in C_p(\lambda)$ and $\tau \in C_p(\lambda^\prime)$}. 
\item \emph{*Under what conditions, $0$-dimensional Persistent Homology of $n$ sample points in Euclidean space is consistent to recover connectivity of a probability density super-level set}?  (Hartigen, J. A. ``Statistical theory in clustering", Journal of Classification, 2: 63-76, 1985)
\end{enumerate}

\item *{\em Niyogi-Smale-Weinberger Consistency Theorem.} The following theorem establishes the homology consistency of C\v{e}ch complex of discrete samples uniformly from a compact submanifold of $\R^d$. 
\paragraph{Theorem.}{} Let $M$ be a compact submanifold of $\mathbb{R}^{d}$ with condition number $\tau$ and let $\left\{x_{1}, \ldots, x_{n}\right\}$ be a set of points drawn from $M$ according to the volume measure. Fix $0<\epsilon<\frac{\tau}{2}$. Then if
$$
n>\beta_{1}\left(\log \left(\beta_{2}\right)+\log \left(\frac{1}{\delta}\right)\right),
$$
there is a homotopy equivalence
$$
\bigcup_{z \in\left\{x_{1}, \ldots, x_{n}\right\}} B_{\epsilon}(z) \simeq M
$$
between the union of balls and $M$ (and in particular the homology groups coincide) with probability $>1-\delta$.
Here
$$
\beta_{1}=\frac{\operatorname{vol}(M)}{\cos ^{d}\left(\theta_{1}\right) \operatorname{vol}\left(B_{\frac{\epsilon}{4}}^{d}\right)}
$$
and
$$
\beta_{2}=\frac{\operatorname{vol}(M)}{\cos ^{d}\left(\theta_{2}\right) \operatorname{vol}\left(B^d_{\frac{\epsilon}{8}} \right)},
$$
where $\theta_{1}=\arcsin \left(\frac{\epsilon}{8 \tau}\right), \theta_{2}=\arcsin \left(\frac{\epsilon}{16 \tau}\right)$, and $\operatorname{vol}\left(B_{r}^{d}\right)$ denotes the volume of the
$d$-dimensional ball of radius $r$.
\emph{Can you read the paper and sketch the main idea for the proof? What about the noisy samples near the submanifold? }
\paragraph{Reference:} Partha Niyogi, Stephen Smale, and Shmuel Weinberger. \emph{Finding the homology of submanifolds with high confidence from random samples}. Discrete Comput. Geom. 39(1-3): 419-441, 2008.
%\item {\em Mapper}: 
\end{enumerate}

\end{document}


