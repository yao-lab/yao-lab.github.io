\documentclass[11pt]{article}

\usepackage{amssymb}
\usepackage{amsmath}
\usepackage{graphicx}
\usepackage{hyperref}

\def\N{{\mathbb N}}
\def\NN{{\mathcal N}}
\def\R{{\mathbb R}}
\def\E{{\mathbb E}}
\def\rank{{\mathrm{rank}}}
\def\tr{{\mathrm{trace}}}
\def\P{{\mathrm{Prob}}}
\def\sign{{\mathrm{sign}}}
\def\diag{{\mathrm{diag}}}

\setlength{\oddsidemargin}{0.25 in}
\setlength{\evensidemargin}{-0.25 in}
\setlength{\topmargin}{-0.6 in}
\setlength{\textwidth}{6.5 in}
\setlength{\textheight}{8.5 in}
\setlength{\headsep}{0.75 in}
\setlength{\parindent}{0.25 in}
\setlength{\parskip}{0.1 in}

\newcommand{\lecture}[4]{
   \pagestyle{myheadings}
   \thispagestyle{plain}
   \newpage
   \setcounter{page}{1}
   \setcounter{section}{0}
   \noindent
   \begin{center}
   \framebox{
      \vbox{\vspace{2mm}
    \hbox to 6.28in { {\bf A Mathematical Introduction to Data Science \hfill #4} }
       \vspace{6mm}
       \hbox to 6.28in { {\Large \hfill #1  \hfill} }
       \vspace{6mm}
       \hbox to 6.28in { {\it Instructor: #2\hfill #3} }
      \vspace{2mm}}
   }
   \end{center}
   \markboth{#1}{#1}
   \vspace*{4mm}
}


\begin{document}

\lecture{Homework 6. Manifold Learning}{Yuan Yao}{Due: 2 weeks later}{March 20, 2023}

The problem below marked by $^*$ is optional with bonus credits. % For the experimental problem, include the source codes which are runnable under standard settings. 

\begin{enumerate}

\item {\em Order the faces:} The following dataset contains 33 faces of the same person ($Y\in \R^{112\times 92\times33}$) in different angles,

\url{https://yao-lab.github.io/data/face.mat}

You may create a data matrix $X\in \R^{n\times p}$ where $n=33,p=112\times92=10304$ (e.g. {\texttt{X=reshape(Y,[10304,33])';}} in matlab).

\begin{enumerate}
\item Explore the MDS-embedding of the 33 faces on top two eigenvectors: order the faces according to the top 1st eigenvector and visualize your results with figures. 
\item Explore the ISOMAP-embedding of the 33 faces on the $k=5$ nearest neighbor graph and compare it against the MDS results. Note: you may try Tenenbaum's Matlab code \\
{\url{https://yao-lab.github.io/data/isomapII.m}}
\item Explore the LLE-embedding of the 33 faces on the $k=5$ nearest neighbor graph and compare it against ISOMAP. Note: you may try the following Matlab code \\
{\url{https://yao-lab.github.io/data/lle.m}}
\end{enumerate}

\item {\em Manifold Learning}: The following codes by Todd Wittman contain major manifold learning algorithms talked on class.

%\url{http://www.math.pku.edu.cn/teachers/yaoy/Spring2011/matlab/mani.m}
\url{http://math.stanford.edu/~yuany/course/data/mani.m}

Precisely, eight algorithms are implemented in the codes: MDS, PCA, ISOMAP, LLE, Hessian Eigenmap, Laplacian Eigenmap, Diffusion Map, and LTSA. 
The following nine examples are given to compare these methods,
\begin{enumerate}
\item Swiss roll;
\item Swiss hole;
\item Corner Planes;
\item Punctured Sphere;
\item Twin Peaks;
\item 3D Clusters;
\item Toroidal Helix;
\item Gaussian;
\item Occluded Disks.
\end{enumerate}
Run the codes for each of the nine examples, and analyze the phenomena you observed. 

\subitem *Moreover if possible, play with t-SNE using scikit-learn manifold package:

\url{http://scikit-learn.org/stable/modules/generated/sklearn.manifold.TSNE.html}

or any other implementations collected at 

\centering{\url{http://lvdmaaten.github.io/tsne/} }

%\subitem* Moreover if possible, make an implementation of Vector Diffusion Map (Laplacian) to reconstruct Swiss roll etc., and compare it against others. Note that LTSA looks for the following coordinates,
%\[ \min_{Y}  \sum_{i\sim j} \| y_i - U_i U_j^T y_j \|^2 \]
%where $U_i$ is a local PCA basis for tangent space at point $x_i\in \R^p$; in a contrast, vector connection Laplacian looks for:
%\[ \min_{Y}  \sum_{i\sim j} \| y_i - O_{ij} y_j \|^2, \ \ \ O_{ij} = \arg \min_O \|U_i - O_{ij} U_j \|^2 \]
%where $O_{ij}$ is an isometry (parallel transportation) from the tangent space at point $x_j$ to the tangent space at point $x_i$, whose solution is given by the lower eigenvectors of the following vector connection Laplacian
%\[
%L_{ij} = \left\{
%\begin{array}{lr}
%-w_{ij} O_{ij}, & i\neq j \\
%\sum_{j} w_{ij} I, & i=j.
%\end{array}
%\right.
%\]
%Here $w_{ij}=w_{ji}$ defines the similarity (neighborhood) between point $x_i$ and $x_j$. For details, please refer to the paper by Amit Singer and Hau-Tieng Wu, \emph{Vector diffusion maps and the connection laplacian}, Comm. Pure Appl. Math. 65 (2012), no. 8, 1067–1144. You may try Hau-Tieng Wu's matlab codes on VDM from the following website
%
%\centering{\url{https://sites.google.com/site/hautiengwu/home/download}}


\item {\em Nystr\"{o}m method:} In class, we have shown that every manifold learning algorithm can be regarded as Kernel PCA on graphs: (1) given $N$ data points, define a neighborhood graph with $N$ nodes for data points; (2) construct a positive semidefinite kernel $K$; (3) pursue spectral decomposition of $K$ to find the embedding (using top or bottom eigenvectors). However, this approach might suffer from the expensive computational cost in spectral decomposition of $K$ if $N$ is large and $K$ is non-sparse, e.g. ISOMAP and MDS. 

To overcome this hurdle, Nystr\"{o}m method leads us to a scalable approach to compute eigenvectors of low rank matrices.  
Suppose that an $N$-by-$N$ positive semidefinite matrix $K\succeq 0$ admits the following block partition
\begin{equation}
K = \left[
\begin{array}{cc}
A & B \\
B^T & C
\end{array}
\right].
\end{equation}
where $A$ is an $n$-by-$n$ block. Assume that $A$ has the spectral decomposition $A=U \Lambda U^T$, $\Lambda=\diag(\lambda_i)$ ($\lambda_1\geq \lambda_2 \geq \ldots \lambda_k > \lambda_{k+1}=\ldots=0$) and $U=[u_1,\ldots,u_n]$ satisfies $U^TU=I$. 
\begin{enumerate}
\item Assume that $K=X X^T$ for some $X=[X_1; X_2] \in \R^{N\times k}$ with the block $X_1\in \R^{n\times k}$. Show that $X_1$ and $X_2$ can be decided by:
\begin{equation}\label{eq:x1}
 X_1 = U_k \Lambda_k^{1/2}, 
\end{equation}
\begin{equation} \label{eq:x2}
X_2 = B^T U_k \Lambda_k^{-1/2}, 
\end{equation}
%\[ X_1 X_1^T = A \Rightarrow X_1 = U_k \Lambda_k^{1/2}, \]
%\[ X_1 X_2^T = B \Rightarrow X_2 = B^T U_k \Lambda_k^{-1/2}. \]  
where $U_k=[u_1, \ldots, u_k]$ consists of those $k$ columns of $U$ corresponding to top $k$ eigenvalues $\lambda_i$ ($i=1,\ldots,k$).  
\item Show that for general $K\succeq 0$, one can construct an approximation from \eqref{eq:x1} and \eqref{eq:x2}, 
\begin{equation}
\hat{K} = \left[
\begin{array}{cc}
A & B \\
B^T & \hat{C} 
\end{array}
\right].
\end{equation}
where $A=X_1 X_1^T$, $B= X_1 X_2^T$, and $\hat{C}=X_2 X_2^T =   B^T A^\dagger B$, $A^\dagger$ denoting the Moore-Penrose (pseudo-) inverse of $A$. Therefore $\|\hat{K}-K\|_F =\|C- B^T A^\dagger B\|_F$. Here the matrix $C-B^T A^\dagger B=:K/A$ is called the (generalized) \emph{Schur Complement} of $A$ in $K$.
\item Explore Nystr\"{o}m method on the Swiss-Roll dataset (\url{http://yao-lab.github.io/data/swiss_roll_data.mat} contains 3D-data X; \url{http://yao-lab.github.io/data/swissroll.m} is the matlab code) with ISOMAP. To construct the block $A$, you may choose either of the following: 
\subitem $n$ random data points; 
\subitem *$n$ landmarks as minimax $k$-centers (\url{https://yao-lab.github.io/data/kcenter.m}); \\
Some references can be found at: 
\subitem[dVT04] Vin de Silva and J. B. Tenenbaum, ``Sparse multidimensional scaling using landmark points", 2004, downloadable at \url{http://pages.pomona.edu/~vds04747/public/papers/landmarks.pdf};  
\subitem[P05] John C. Platt, ``FastMap, MetricMap, and Landmark MDS are all Nystr\"{o}m Algorithms", 2005, downloadable at \url{http://research.microsoft.com/en-us/um/people/jplatt/nystrom2.pdf}. 
\item *Assume that $A$ is invertible, show that  
\[ \det(K) = \det(A)\cdot \det(K/A), \]
\item *Assume that $A$ is invertible, show that 
\[ \rank(K)=\rank(A)+\rank(K/A). \]
\item *Can you extend the identities in (c) and (d) to the case of noninvertible $A$? A good reference can be found at, 
\subitem[Q81] Diane V. Quellette, ``Schur Complements and Statistics", Linear Algebra and Its Applications, 36:187-295, 1981. \url{http://www.sciencedirect.com/science/article/pii/0024379581902329}

\end{enumerate}

%$$
%\begin{array}{ll}{{\text { minimize }}} & {\|L\|_{*}+\lambda\|S\|_{1}} \\ {\text { subject to }} & {L+S=M}\end{array}
%$$

%\item {\em James Stein Estimator for $p=1$:} 
%
%From Theorem 3.1 in the lecture notes, we know that MLE $\hat{\mu} = Y$ is admissible when $p=1 \text{ or } 2$. However if we use SURE to calculate the risk of James Stein Estimator,
%\[
%	R(\hat{\mu}^{\text{JS}},\mu) = \E U(Y) 
%	= p - \E_\mu \frac{(p-2)^2}{\lVert Y \rVert^2}
%	< p = R(\hat{\mu}^{\text{MLE}},\mu)
%\]
%it seems that for $p=1$ James Stein Estimator should still has lower risk than MLE for any $\mu$.
% Explain what violates the above calculation for $p=1$.
%

\end{enumerate}

\end{document}


