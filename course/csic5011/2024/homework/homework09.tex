\documentclass[11pt]{article}

\usepackage{amssymb}
\usepackage{amsmath}
\usepackage{graphicx}
\usepackage{hyperref}
\usepackage{tikz}	% for drawing stuff
\usepackage{xcolor}    % for \textcolor{}
\usepackage{readarray} % for \getargsC{}

\def\N{{\mathbb N}}
\def\NN{{\mathcal N}}
\def\R{{\mathbb R}}
\def\E{{\mathbb E}}
\def\rank{{\mathrm{rank}}}
\def\tr{{\mathrm{trace}}}
\def\P{{\mathrm{Prob}}}
\def\sign{{\mathrm{sign}}}
\def\diag{{\mathrm{diag}}}
\def\im{{\mathrm{image}}}
\def\ker{{\mathrm{ker}}}

\setlength{\oddsidemargin}{0.25 in}
\setlength{\evensidemargin}{-0.25 in}
\setlength{\topmargin}{-0.6 in}
\setlength{\textwidth}{6.5 in}
\setlength{\textheight}{8.5 in}
\setlength{\headsep}{0.75 in}
\setlength{\parindent}{0.25 in}
\setlength{\parskip}{0.1 in}

\newcommand{\lecture}[4]{
   \pagestyle{myheadings}
   \thispagestyle{plain}
   \newpage
   \setcounter{page}{1}
   \setcounter{section}{0}
   \noindent
   \begin{center}
   \framebox{
      \vbox{\vspace{2mm}
    \hbox to 6.28in { {\bf A Mathematical Introduction to Data Science \hfill #4} }
       \vspace{6mm}
       \hbox to 6.28in { {\Large \hfill #1  \hfill} }
       \vspace{6mm}
       \hbox to 6.28in { {\it Instructor: #2\hfill #3} }
      \vspace{2mm}}
   }
   \end{center}
   \markboth{#1}{#1}
   \vspace*{4mm}
}

\definecolor{darkred}{rgb}{0.64,0,0}
\definecolor{darkcyan}{rgb}{0,0.55,0.55}
\newcommand{\rowcolor}[1]{\textcolor{darkred}{#1}}
\newcommand{\columncolor}[1]{\textcolor{darkcyan}{#1}}

% Normal-form game
% \nfgame{T B L R RTL RTR RBL RBR CTL CTR CBL CBR}
\newcommand{\nfgame}[1]{%
\getargsC{#1}
\begin{tikzpicture}[scale=0.65]

\node (RT) at (-2,1) [label=left:\rowcolor{\argi}] {};
\node (RB) at (-2,-1) [label=left:\rowcolor{\argii}] {};
\node (CL) at (-1,2) [label=above:\columncolor{\argiii}] {};
\node (CR) at (1,2) [label=above:\columncolor{\argiv}] {};

\node (RTL) at (-1.4,0.6) {\rowcolor{\argv}}; % top/left row player payoff etc.
\node (RTR) at (0.6,0.6) {\rowcolor{\argvi}};
\node (RBL) at (-1.4,-1.4) {\rowcolor{\argvii}};
\node (RBR) at (0.6,-1.4) {\rowcolor{\argviii}};

\node (CTL) at (-0.6,1.4) {\columncolor{\argix}};
\node (CTR) at (1.4,1.4) {\columncolor{\argx}};
\node (CBL) at (-0.6,-0.6) {\columncolor{\argxi}};
\node (CBR) at (1.4,-0.6) {\columncolor{\argxii}};

\draw[-,very thick] (-2,-2) to (2,-2);
\draw[-,very thick] (-2,0) to (2,0);
\draw[-,very thick] (-2,2) to (2,2);
\draw[-,very thick] (-2,-2) to (-2,2);
\draw[-,very thick] (0,-2) to (0,2);
\draw[-,very thick] (2,-2) to (2,2);
\draw[-,very thin] (-2,2) to (0,0);
\draw[-,very thin] (0,0) to (2,-2);
\draw[-,very thin] (-2,0) to (0,-2);
\draw[-,very thin] (0,2) to (2,0);
\end{tikzpicture}}

\begin{document}

\lecture{Homework 9. Combinatorial Hodge Theory}{Yuan Yao}{Due: 1 weeks later}{Apr. 24, 2024}

The problem below marked by $^*$ is optional with bonus credits. % For the experimental problem, include the source codes which are runnable under standard settings. 

\begin{enumerate}
\item {\em HodgeRank:} Download the HodgeRank matlab codes and unzip: 

\url{https://yao-lab.github.io/publications/BatchHodge.zip}

which contains two subfolders. 
\begin{itemize}
\item \url{./data/}: file \texttt{incomp.mat} contains a 1668-by-2 matrix, collecting 1668 pairwise comparisons among 16 video items, with the first column index preferred to the second ones;
\item \url{./code/}: file \texttt{Hodgerank.m} is the Hodge decomposition of such pairwise comparison data. 
\end{itemize}

Run the following command
\begin{quote} \begin{verbatim} 
>> load data/incomp.mat
>> cd code
>> [score,totalIncon,harmIncon] = Hodgerank(incomp)
\end{verbatim} \end{quote}

You will return with global ranking scores (generalized Borda count) in \texttt{score}, a 16-by-4 matrix as scores of 16 videos in 4 models:  
\begin{quote} \begin{verbatim} 
	model1: Uniform noise model, Y_hat(i,j) = 2 pij -1
	model2: Bradley-Terry, Y_hat(i,j) = log(abs(pij+eps))-log(abs(1-pij-eps))
	model3: Thurstone-Mosteller, Y_hat(i,j) ~ norminv(abs(1-pij-eps));
        model4: Arcsin, Y_hat4(i,j) = asin(2*pij-1)
\end{verbatim} \end{quote}
and two inconsistency measurements (total inconsistency \texttt{totalIncon} = harmonic inconsistency \texttt{harmIncon} + triangular inconsistency). The following ratio: 
\begin{quote} \begin{verbatim}
>> harmIncon/totalIncon
\end{verbatim} \end{quote}
measures the percentage of harmonic inconsistency in the total inconsistency (residue).

Moreover, \emph{can you compute the HodgeRank for the weblink data?} For example, the following dataset contains Chinese (mainland) University Weblink during 12/2001-1/2002,

\url{https://github.com/yao-lab/yao-lab.github.io/blob/master/data/univ_cn.mat}

\emph{compute the HodgeRank scores and compare them against PageRank and HITs etc.}

\paragraph{Reference:}
\begin{itemize}
\item Xiaoye Jiang, Lek-Heng Lim, Yuan Yao and Yinyu Ye. \emph{Statistical Ranking and Combinatorial Hodge Theory}.
Mathematical Programming, Volume 127, Number 1, Pages 203-244, 2011.
\item Qianqian Xu, Qingming Huang, Tingting Jiang, Bowei Yan, Weisi Lin and Yuan Yao, \emph{HodgeRank on Random Graphs for Subjective Video Quality  
Assessment}, IEEE Transaction on Multimedia, vol. 14, no. 3, pp. 844-857, 2012.
\end{itemize}


\item {\em Hodge Decomposition in Linear Algebra.} For inner product spaces $X$, $Y$, and $Z$, consider	
	\[
	X \xrightarrow{A} Y \xrightarrow{B} Z.
	\]
	and $\Delta =A A^T + B^T B: Y\to Y$ where $A^T$ ($B^T$) is the adjoint of $A$ ($B$) such that $\langle Ax,y\rangle=\langle x, A^T y\rangle$ ($\langle y,B^Tz\rangle=\langle By, z\rangle$), respectively.
	Show that if the following composition vanishes,
	\[ B \circ A = 0,\]
	then $\ker(\Delta) = \ker(A^T) \cap \ker(B)$ and the following \emph{orthogonal} decomposition holds
	\[ Y = \im (A) + \ker(\Delta) + \im (B^T). \]
	
	%Note: $\textcolor{blue}{\ker(B)/\im(A)}\simeq \ker(\Delta)$ is the (real) (co)-homology group ($\R\to$ rings; vector spaces$\rightarrow$module).

\item {\em *Hodge Decomposition of the Prisoner's Dilemma Game}: Consider the normal form game of the Prisoner's Dilemma, whose row and column players can play $C$ (cooperate) or $D$ (defect) and receive the payoffs, respectively (as in the table). Show that \emph{the Hodge Decomposition of its game flow is a potential game}.   


\begin{itemize}
\item Prisoner's Dilemma:
\begin{center}
\nfgame{C D C D $3$ $0$ $5$ $1$ $3$ $5$ $0$ $1$} 
%\qquad
%\nfgame{T B L R $1$ $2$ $3$ $4$ $5$ $6$ $7$ $8$} \\[10pt]
\end{center}
\item Putin's War against Nato:
\begin{center}
\nfgame{Attack Quit Expand Stop $-10$ $2$ $-20$ $1$ $-10$ $1$ $5$ $2$} 
%\qquad
%\nfgame{T B L R $1$ $2$ $3$ $4$ $5$ $6$ $7$ $8$} \\[10pt]
\end{center}
\end{itemize}

\paragraph{Reference:} Ozan Candogan, Ishai Menache, Asuman Ozdaglar, and Pablo A. Parrilo. \emph{Flows and Decompositions of Games: Harmonic and Potential Games}.
Mathematics of Operations Research, 36(3): 474 - 503, 2011.

\end{enumerate}

\end{document}


