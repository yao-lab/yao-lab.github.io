\documentclass[11pt]{article}

\usepackage{amssymb}
\usepackage{amsmath}
\usepackage{graphicx}
\usepackage{hyperref}

\def\A{{\mathcal A}}
\def\Ln{{\mathcal L}}
\def\N{{\mathbb N}}
\def\NN{{\mathcal N}}
\def\R{{\mathbb R}}
\def\E{{\mathbb E}}
\def\rank{{\mathrm{rank}}}
\def\tr{{\mathrm{trace}}}
\def\P{{\mathrm{Prob}}}
\def\sign{{\mathrm{sign}}}
\def\diag{{\mathrm{diag}}}

\setlength{\oddsidemargin}{0.25 in}
\setlength{\evensidemargin}{-0.25 in}
\setlength{\topmargin}{-0.6 in}
\setlength{\textwidth}{6.5 in}
\setlength{\textheight}{8.5 in}
\setlength{\headsep}{0.75 in}
\setlength{\parindent}{0.25 in}
\setlength{\parskip}{0.1 in}

\newcommand{\lecture}[4]{
   \pagestyle{myheadings}
   \thispagestyle{plain}
   \newpage
   \setcounter{page}{1}
   \setcounter{section}{0}
   \noindent
   \begin{center}
   \framebox{
      \vbox{\vspace{2mm}
    \hbox to 6.28in { {\bf A Mathematical Introduction to Data Science \hfill #4} }
       \vspace{6mm}
       \hbox to 6.28in { {\Large \hfill #1  \hfill} }
       \vspace{6mm}
       \hbox to 6.28in { {\it Instructor: #2\hfill #3} }
      \vspace{2mm}}
   }
   \end{center}
   \markboth{#1}{#1}
   \vspace*{4mm}
}


\begin{document}

\lecture{Homework 7. Markov Chains on Graphs and Spectral Theory}{Yuan Yao}{Due: 2 weeks later}{March 27, 2023}

The problem below marked by $^*$ is optional with bonus credits. % For the experimental problem, include the source codes which are runnable under standard settings. 

\begin{enumerate}

\item {\em PageRank}:  The following dataset contains Chinese (mainland) University Weblink during 12/2001-1/2002,

\url{https://github.com/yao-lab/yao-lab.github.io/blob/master/data/univ_cn.mat}

where $\tt{rank\_{}cn}$ is the research ranking of universities in that year, $\tt{univ\_{}cn}$ contains the webpages of universities, and $\tt{W\_{}cn}$ is the link matrix from university
$i$ to $j$. 
 
\begin{enumerate}
\item Compute PageRank with Google's hyperparameter $\alpha=0.85$;
\item Compute HITS authority and hub ranking using SVD of the link matrix; 
\item Compare these rankings against the research ranking (you may consider Kendall's $\tau$ distance -- as the number of pairwise mismatches between two orders -- to compare different rankings);  
\item Compute extended PageRank with various hyperparameters $\alpha\in (0,1)$, investigate its effect on ranking stability. 
\end{enumerate} 

For your reference, an implementation of PageRank and HITs can be found at 

\url{https://github.com/yao-lab/yao-lab.github.io/blob/master/data/pagerank.m}

\item {\em Perron Theorem:} Assume that $A>0$. Consider the following optimization problem:
\begin{eqnarray*}
& & \max \delta \\
& s.t. & Ax\geq\delta x \\
&& x\geq 0 \\
&& x \neq 0.
\end{eqnarray*}
Let $\lambda^*$ be optimal value with $\nu^*\geq0,\quad 1^T\nu^*=1$, and $A\nu^*\geq\lambda^*\nu^*$. Show that
\begin{enumerate}
\item $A\nu^*=\lambda^*\nu^*$, i.e. $(\lambda^*,\nu^*)$ is an eigenvalue-eigenvector pair of $A$;\\
\item  $\nu^*>0$;\\
\item[*(c)]$\lambda^*$ is unique and $\nu^*$ is unique;\\
\item[*(d)] For other eigenvalue $\lambda\quad(\lambda z=Az\quad when\quad z\neq0)$, $|\lambda|<\lambda^*$.
\end{enumerate}


\item {\em *Absorbing Markov Chain:} 

Let $P$ be a row Markov matrix on $n+1$ states with non-absorbing state $\{1,\ldots,n\}$ and absorbing state $n+1$. Then $P$ can be partitioned into 
\[
	P=\left[
	\begin{array}{cc} 
	Q & R \\ 
	0 & 1 
	\end{array} \right]
\]
Assume that $Q$ is primitive. Let $N(i,j)$ be the expected number of jumps starting from nonabsorbent state $i$ and hitting state $j$, before reaching the absorbing state $n+1$. Show
that
\begin{enumerate}
\item $N(i,i) = 1 + \sum_k N(i,k) Q(k,i)$, for $i=1,\ldots,n$;
\item $N(i,j) = \sum_k N(i,k) Q(k,j)$, for $i\neq j$;
\item These identities together imply that $N=(I-Q)^{-1}$, called the fundamental matrix; 
\item Show that the probability of absorption from state $i$, $B(i)$ ($i=1\ldots,n$), is given by $B=NR$.   
\end{enumerate}

\item {\em Spectral Bipartition}: Consider the 374-by-475 matrix $X$ of character-event for A Dream of Red Mansions, e.g. in the Matlab format 

\url{https://github.com/yuany-pku/dream-of-the-red-chamber/blob/master/HongLouMeng374.txt}

\noindent with a readme file:

\url{https://github.com/yuany-pku/dream-of-the-red-chamber/blob/master/README.md}

Construct a weighted adjacency matrix for character-cooccurance network $A=X X^T$. Define the degree matrix $D=\diag(\sum_j A_{ij})$. Check if the graph is connected. 
If you are not familiar with this novel and would like to work on a different network, you may consider the Karate Club Network: 

\url{https://github.com/yao-lab/yao-lab.github.io/blob/master/data/karate.mat} 

\noindent that contains a 34-by-34 adjacency matrix. 

\begin{enumerate}
\item Find the second smallest generalized eigenvector of $L=D-A$, i.e. $(D-A)f = \lambda_2 f$ where $\lambda_2>0$;
\item Sort the nodes (characters) according to the ascending order of $f$, such that $f_1\leq f_2 \leq \ldots \leq f_n$, and construct the subset $S_i = \{1,\ldots, i\}$;
\item Find an optimal subset $S^\ast$ such that the following is minimized  
\[ \alpha_f = \min_{S_i} \left\{ \frac{|\partial S_i|} {\min(|S_i|, |\bar{S}_i|)} \right\}\]  
where $|\partial S_i|=\sum_{x\sim y, x\in S_i, y\in \bar{S}_i} A_{xy}$ and $|S_i|=\sum_{x\in S_i} d_x = \sum_{x\in S_i, y} A_{xy}$.  
\item Check if $\lambda_2 > \alpha_f$;
\item Quite often people find a suboptimal cut by $S^+=\{i: f_i \geq 0\}$ and $S^-=\{i: f_i <0\}$. Compute its Cheeger ratio
\[ h_{S^+} =  \frac{|\partial S^+|} {\min(|S^+|, |S^-|)} \]
and compare it with $\alpha_f$, $\lambda_2$. 
\item You may further recursively bipartite the subgraphs into two groups, which gives a recursive spectral bipartition. 
\end{enumerate} 

\item {\em Degree Corrected Stochastic Block Model (DCSBM)}: A random graph is generated from a DCSBM with respect to partition $\Omega=\{\Omega_k: k=1,\ldots,K\}$ if its adjacency matrix $A \in \{0,1\}^{N\times N}$ has the following expectation

\[ \E [A] = \A = \Theta Z B Z^T \Theta\]

where $Z^{N \times k}$ has row vectors $\thickspace \in {\{0,1\}}^K$ as the block membership function $z:V\to \Omega$,
\begin{equation*}
z_{ik}=
 \begin{cases}
    1,   &  i\in \Omega_k, \\
    0,   &  otherwise.
 \end{cases}
\end{equation*}
and $\Theta = \diag(\theta_i)$ is the expected degree satisfying,
\begin{equation*}
  \sum_{i\in \Omega_k}\theta_i=1, \ \ \ \forall k=1,\ldots,K.
\end{equation*}

The following matlab codes simulate a DCSBM of $n K$ nodes, written by Kaizheng Wang, 

\url{https://github.com/yao-lab/yao-lab.github.io/blob/master/data/DCSBM.m} 

Construct a DCSBM yourself, and simulate random graphs of 10 times. Then try to compare the following two spectral clustering methods in finding the $K$ blocks (communities).  
\begin{enumerate}
\item[Alg. A]
\subitem[1] Compute the \emph{top} $K$ generalized eigenvector $$(D - A) \phi_i = \lambda_i D \phi_i,$$ construct a $K$-dimensional embedding of $V$ using $\Phi^{N\times K} = [\phi_1, \ldots, \phi_K]$; 
\subitem[2] Run $k$-means algorithm (call {\tt{kmeans}} in matlab) on $\Phi$ to find $K$ clusters.
\item[Alg. B]
\subitem[1] Compute the \emph{bottom} $K$ eigenvector of $$\Ln = D^{-1/2}(D-A)D^{-1/2} = U \Lambda U^T,$$ construct an embedding of $V$ using $U^{N\times K}$;
\subitem[2] Normalized the row vectors $u_{i\ast}$ on to the sphere: $\hat{u}_{i\ast} = u_{i\ast}/\|u_{i\ast}\|$;
\subitem[3] Run $k$-means algorithm (call {\tt{kmeans}} in matlab) on $\hat{U}$ to find $K$ clusters.
\end{enumerate} 

You may run it multiple times with a stabler clustering. Suppose the estimated membership function is $\hat{z}: V\to \{1,\ldots,K\}$ in either methods. Compare the performance using mutual information between membership function $z$ and estimate $\hat{z}$,
\begin{equation}
I(z, \hat{z}) = \sum_{s,t=1}^K Prob(z_i = s, \hat{z}_i=t) \log \frac{Prob(z_i=s, \hat{z}_i=t)}{Prob(z_i= s) Prob(\hat{z}_i= t)}.
\end{equation}
For example, 

\url{https://github.com/yao-lab/yao-lab.github.io/blob/master/data/NormalizedMI.m}




%$$
%\begin{array}{ll}{{\text { minimize }}} & {\|L\|_{*}+\lambda\|S\|_{1}} \\ {\text { subject to }} & {L+S=M}\end{array}
%$$

%\item {\em James Stein Estimator for $p=1$:} 
%
%From Theorem 3.1 in the lecture notes, we know that MLE $\hat{\mu} = Y$ is admissible when $p=1 \text{ or } 2$. However if we use SURE to calculate the risk of James Stein Estimator,
%\[
%	R(\hat{\mu}^{\text{JS}},\mu) = \E U(Y) 
%	= p - \E_\mu \frac{(p-2)^2}{\lVert Y \rVert^2}
%	< p = R(\hat{\mu}^{\text{MLE}},\mu)
%\]
%it seems that for $p=1$ James Stein Estimator should still has lower risk than MLE for any $\mu$.
% Explain what violates the above calculation for $p=1$.
%

\item {\em *Directed Graph Laplacian}: Consider the following dataset with Chinese (mainland) University Weblink during 12/2001-1/2002,

\url{https://github.com/yao-lab/yao-lab.github.io/blob/master/data/univ_cn.mat}

where $\tt{rank\_{}cn}$ is the research ranking of universities in that year, $\tt{univ\_{}cn}$ contains the webpages of universities, and $\tt{W\_{}cn}$ is the link matrix from university
$i$ to $j$. 

Define a PageRank Markov Chain
\[  P = \alpha P_0 + (1-\alpha) \frac{1}{n} e e^T, \ \ \ \alpha = 0.85 \]
where $P_0 = D_{out}^{-1} A$. Let $\phi\in \R_+^n$ be the stationary distribution of $P$, i.e. PageRank vector. Define $\Phi = \diag(\phi_i)\in \R^{n\times n}$.
\begin{enumerate}
\item Construct the normalized directed Laplacian 
\[ \mathcal{\vec{L}}=I - \frac{1}{2}(\Phi^{1/2} P \Phi^{-1/2} + \Phi^{-1/2} P^T \Phi^{1/2} ) \]
\item Use the second eigenvector of $\mathcal{\vec{L}}$ to bipartite the universities into two groups, and describe your algorithm in detail;
\item Try to explain your observation through directed graph Cheeger inequality.
\end{enumerate} 
  


\item {\em *Chung's Short Proof of Cheeger's Inequality}: 

Chung's short proof is based on the fact that 
\begin{equation} 
h_G = \inf_{f\neq 0} \sup_{c\in \R} \frac{\sum_{x\sim y} |f(x) - f(y)|}{\sum_x |f(x) -c|d_x} 
\end{equation}
where the supreme over $c$ is reached at $c^*\in median(f(x):x\in V)$. Such a claim can be found in Theorem 2.9 in Chung's monograph, Spectral Graph Theory. In fact, Theorem 2.9 implies that the infimum above is reached at certain function $f$. From here, 
\begin{eqnarray}
\lambda_1 & =& R(f)=\sup_c \dfrac{\sum_{x\sim y}(f(x) - f(y))^2}{\sum_{x}(f(x)-c)^2 d_x},  \\
& \geq  &\dfrac{\sum_{x\sim y}(g(x) - g(y))^2}{\sum_{x}g(x)^2 d_x}, \ \ \ g(x)=f(x)-c\\
& = & \dfrac{(\sum_{x\sim y}(g(x) - g(y))^2)(\sum_{x\sim y}(g(x) +g(y))^2)}{(\sum_{x\in V}g^2(x)d_x)((\sum_{x\sim y}(g(x) +g(y))^2)} \\
& \ge & \dfrac{(\sum_{x\sim y}|g^2(x) - g^2(y)|)^2}{(\sum_{x\in V}g^2(x)d_x)((\sum_{x\sim y}(g(x) +g(y))^2)} , \ \ \textrm{Cauchy-Schwartz Inequality} \\
& \ge & \dfrac{(\sum_{x\sim y}|g^2(x) - g^2(y)|)^2}{2( \sum_{x\in V}g^2(x)d_x)^2} , \ \ \textrm{$(g(x)+g(y))^2\leq 2 (g^2(x)+g^2(y))$} \\
& \ge & \dfrac{h_G^2}{2}.
\end{eqnarray}
Is there any step {\bf wrong} in the reasoning above? If yes, can you remedy it/them?

\end{enumerate}

\end{document}


