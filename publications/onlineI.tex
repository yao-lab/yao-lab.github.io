\documentclass[twoside]{amsart}
\usepackage{amsmath}
\usepackage{amsthm}
\usepackage{amssymb}
\usepackage{mathrsfs}

\theoremstyle{theorem}
\newtheorem*{mainthm}{Main Theorem}
\newtheorem*{thma}{Theorem A}
\newtheorem*{thmb}{Theorem B}
\newtheorem*{thmb1}{Theorem B'}
\newtheorem*{thmc}{Theorem C}
\newtheorem*{lem*}{Lemma}
\newtheorem*{pf}{Proof}


\newtheorem{thm}{Theorem}[section]
\newtheorem{conj}{Conjecture}[section]
\newtheorem{lem}[thm]{Lemma}
\newtheorem{lema}{Lemma A}
\newtheorem{lemb}{Lemma B}
\newtheorem{thmaa}{Theorem A}
\newtheorem{thmbb}{Theorem B}
\newtheorem{prop}[thm]{Proposition}
\newtheorem{cor}[thm]{Corollary}
\newtheorem{cora}[thmaa]{Corollary A}
\newtheorem{corb}[thmbb]{Corollary B}

\theoremstyle{definition}
\newtheorem*{defn}{Definition}
\newtheorem*{defa}{Definition A}
\newtheorem{exmp}{Example}[section]

\theoremstyle{remark}
\newtheorem{rem}[thm]{Remark}
\newtheorem*{rem*}{Remark}
\newtheorem*{note}{Note}
\newtheorem*{exer}{Exercise}

\def\L{{\mathscr L}}
\def\A{{\mathscr A}}
\def\C{{\mathscr C}}
\def\H{{\mathscr H}}
\def\K{{\mathscr K}}
\def\I{{\mathscr I}}
\def\S{{\mathscr S}}
\def\T{{\mathscr T}}
\def\Err{{\mathscr E}}
\def\N{{\mathbb N}}
\def\z{{\mathbf z}}
\def\Prob{{\bf Prob}}
\def\Proj{{\rm Proj}}
\def\grad{{\rm grad}}
\def\ess{{\rm ess}}
\def\supp{{\rm supp}}
\def\span{{\rm span}}
\def\O{{\mathscr O}}
\def\amax{{\mu_{\max}}}
\def\amin{{\mu_{\min}} }

\newcommand{\DS}{\displaystyle}

\def\Q{{\mathbb Q}}        % rationals
\def\Z{{\mathbb Z}}        % integers
\def\R{{\mathbb R}}        % reals
\def\Rn{{\R^{n}}}          % product of n copies of reals

\def\P{{\mathbb P}}        % probability
\def\E{{\mathbb E}}        % expectation
\def\1{{\mathbf 1}}        % indicator
\def\var{{\mathop{\mathbf Var}}}    % variance

\def\L{{\mathscr L}}     % L, as in L^2

\def\<{{\langle}}
\def\>{{\rangle}}

\setlength{\oddsidemargin}{0.25 in}
\setlength{\evensidemargin}{-0.25 in}
\setlength{\topmargin}{-0.6 in}
\setlength{\textwidth}{6.5 in}
\setlength{\textheight}{8.5 in}
\setlength{\headsep}{0.75 in}
%\setlength{\parindent}{0 in}
\setlength{\parskip}{0.1 in}

\begin{document}

\title{Online Learning Algorithms}
\thanks{The authors were supported by NSF grant 0325113.}

\author{Steve Smale}
\address{Steve Smale, Toyota Technological Institute at Chicago, 1427 East 60th Street, Chicago, IL
60637.}
\email{smale@math.berkeley.edu}

\author{Yuan Yao}
\address{Yuan Yao, Department of Mathematics, University of California at Berkeley, Berkeley, CA 94720.}
\curraddr{Toyota Technological Institute at Chicago, 1427 East 60th Street, Chicago, IL
60637.}
\email{yao@math.berkeley.edu}

\date{Submitted to \emph{Foundations of Computational Mathematics} on October 25, 2004. Revised on May 1, 2005.}

\keywords{Online Learning, Stochastic Approximation, Regularization, Reproducing Kernel Hilbert Spaces}

\subjclass[2000]{62L20, 68Q32, 68T05}

\maketitle

\begin{abstract}
In this paper, we study an online learning algorithm in Reproducing Kernel Hilbert Spaces (RKHS) and general Hilbert spaces.
We present a general form of the stochastic gradient method to minimize a quadratic potential function by an
independent identically distributed (i.i.d.) sample sequence, and show a probabilistic upper bound for its convergence.
\end{abstract}

\bigskip

%\setcounter{section}{-1}


%%%%%%%%%%%%%%%%%%%%%%%%%%%%%%%%%%%%%%%%%%%%%%%%%%%%%%%%%%%%%%%%%%%%%%%%%%%%%%%%%%%%%%%%%%%%

\section{Introduction}

%%%%%%%%%%%%%%%%%%%%%%%%%%%%%%%%%%%%%%%%%%%%%%%%%%%%%%%%%%%%%%%%%%%%%%%%%%%%%%%%%%%%%%%%%%%%

Consider learning from examples $(x_t, y_t)\in X\times \R$ ($t\in \N$), drawn at random
from a probability measure $\rho$ on
$X\times \R$. For $\lambda>0$, one wants to approximate the function $f^\ast_\lambda$ minimizing over $f\in \H$ the
quadratic functional
\[ \int_{X\times Y} (f(x)-y)^2 d \rho + \lambda \| f \|^2_\H, \]
where $\H$ is some Hilbert space. In this paper a scheme for doing
this is given by using one example at a time $t$ to update to
$f_{t}$ the current hypothesis $f_{t-1}$ which depends only on the
previous examples. The scheme chosen here is based on the
stochastic approximation of the gradient of the quadratic
functional of $f$ displayed above, and takes an especially simple
form in the setting of a ``Reproducing Kernel Hilbert Space''.
Such a stochastic approximation procedure was firstly proposed by
Robbins and Monro \cite{RobMon51}, and as well Kiefer and Wolfowitz
\cite{KieWol52}. Its convergence and asymptotic rates have
been widely studied (e.g. see \cite{RobSie71}, \cite{Duflo97},
\cite{Benaim99}, \cite{Gyofi80}, \cite{Tadic04}, or
\cite{GyoKohKrzWal02} and reference therein). For more background
on stochastic algorithms see for example \cite{BerTsi96},
\cite{Duflo96}, \cite{KusYin03}. The main goal in our development
of the algorithm is to give error estimates which characterize in
probability the distance of our updated hypothesis to
$f^\ast_\lambda$ (and eventually the ``regression function'' of
$\rho$). By choosing a quadratic functional to optimize one is
able to give a deeper understanding of this \emph{online learning}
phenomenon.

In contrast, in the more common setting for Learning Theory the
learner is presented with the whole set of examples in one batch.
One may call this type of work as \emph{batch learning}.

The organization of this paper is as follows. Section 2 presents
an online learning algorithm in Reproducing Kernel Hilbert Spaces
(RKHS) and states Theorem A for a probabilistic upper bound on
initial error and sample error. Section 3 presents a general form
of the stochastic gradient method in Hilbert spaces and Theorem B,
together with a derivation of Theorem A from Theorem B. Section 4
gives the proof of Theorem B and the various bounds appearing in
Section 3. Section 5 compares our results with the case of ``batch
learning''. Section 6 discusses the Adaline or Widrow-Hoff algorithm and related
works. Appendix A collects some estimates used
throughout the paper, Appendix B presents a generalized Bennett's
inequality for independent sums in Hilbert spaces.

The authors would like to acknowledge Peter Bartlett and Pierre Tarr\`es for their suggestions on stepsize rate; Yifeng Yu and
Krishnaswami Alladi for their helpful discussions on proving the Main Analytic Estimate (Lemma A.1) and Lemma A.2; Iosif Pinelis and
Yiming Ying for their pointing out the Lemma B.1 and helpful comments. We thank the reviewers for many suggestions.
We also thank Ding-Xuan Zhou, David McAllester, Adam Kalai, Gang Liang, Leon Bottou
and especially, Tommy Poggio, for many helpful discussions.

\subsection{Notation}
Let $X$ be a closed subset of $\R^n$, $Y=\R$ and $Z=X\times Y$.
Let $\rho$ be a probability measure on $Z$, $\rho_{X}$ be the induced marginal probability measure on $X$,
and $\rho_{Y|x}$ be the conditional probability measure on $Y$ with respect to $x\in
X$. Define $f_\rho:X\to Y$ by
\[ f_\rho (x) = \int_Y y d \rho_{Y|x}, \]
the \emph{regression function of $\rho$}. In other words, for each
$x\in X$, $f_\rho(x)$ is the average of $y$ with respect to
$\rho_{Y|x}$. Let $\L^2_{\rho_X}(X)$ be the Hilbert space of
square integrable functions with respect to $\rho_X$, and denoted
by $\L^2_\rho(X)$ for simplicity. In the sequel $\|\ \|_\rho$
denotes the norm in $\L^2_\rho(X)$ and $\|\ \|_\infty$ denotes the
supreme norm with respect to $\rho_X$ (\emph{i.e.}
$\|f\|_\infty=\ess\sup_{\rho_X} |f(x)|$). We assume that
$\|f_\rho\|_\infty <\infty$ and $f_\rho \in \L^2_{\rho}(X)$. Finally we
use such a convention on the summation and product sign: as
the index $m>n$, let $\sum_{i=m}^n x_i =0$ and $\prod_{i=m}^n x_i=1$, for any summand $x_i$.

\section{An Online Learning Algorithm in RKHS}

Let $K:X\times X\to \R$ be a \emph{Mercer kernel}, i.e. a
continuous symmetric real function which is \emph{positive
semi-definite} in the sense that $\sum_{i,j=1}^l c_i c_j K(x_i,
x_j)\geq 0$ for any $l\in \N$ and any choice of $x_i\in X$ and
$c_i \in \R$ ($i=1,\ldots,l$). Note $K(x,x)\geq 0$ for all $x$. In
the following $\|\cdot \|$ and $\<,\>$ denote the Euclidean norm
and the Euclidean inner product in $\R^n$ respectively.We give two
typical examples of Mercer kernels. The first is the Gaussian
kernel $K:\R^n\times \R^n\to \R$ defined by $K(x,x^\prime)=
\exp(-\|x-x^\prime\|^2/c^2)$ ($c>0$). The second is the linear
kernel $K:\R^n\times \R^n\to \R$ defined by
$K(x,x^\prime)=\<x,x^\prime\>+1$. The restriction of these
functions on $X\times X$ will induce the corresponding kernels on
subsets of $\R^n$.

Let $\H_K$ be the Reproducing Kernel Hilbert Space (RKHS)
associated with a Mercer kernel $K$. Recall the definition as
follows. Consider the vector space $V_K$ generated by $\{K_t:t\in
X\}$, i.e. all the finite linear combinations of $K_t$, where for
each $t\in X$, the function $K_t:X\to \R$ is defined by
$K_t(x)=K(x,t)$. An inner product $\<\ ,\ \>_K$ on this vector
space can be defined as the unique linear extension of
$\<K_x,K_{x'}\>_K:=K(x,x')$ and its induced norm\footnote{Note
that the zero set $V_0=\{f\in V_K: \|f\|_K=0\}=\{0\}$, whence
$\|\cdot\|_K$ is in fact a norm. To see this, by the reproducing
property and Cauchy-Schwartz inequality,
$$ \|f\|_K=0\Rightarrow |f(t)|=|\<f,K_t\>_K| \leq \|f\|_K \|K_t\| =0, \ \forall t\in X\Rightarrow f= 0, $$
which implies $V_0=\{0\}$. } is $\|f\|_K = \sqrt{\<f,f\>_K}$ for
each $f\in V_K$. Let $\H_K$ be the completion of this inner
product space $V_K/V_0$. It follows that for any $f\in \H_K$,
$f(x)=\<f,K_x\>_K$ ($x\in X$). This is often called as the
\emph{reproducing property} in literature. Define a linear map
$L_K:\L^2_\rho(X)\to \H_K$ by $L_K(f)(x)= \int_X K(x,t) f(t) d
\rho_X$. The operator $L_K + \lambda I:\H_K \to \H_K$ is an
isomorphism if $\lambda>0$ (endomorphism if $\lambda\geq 0$),
where $L_K:\H_K \to \H_K$ is the restriction of $L_K:\L^2_\rho(X)
\to \H_K$.

\bigskip

Given a sequence of examples $z_t=(x_t,y_t)\in X\times Y$ ($t\in \N$), our online learning algorithm in RKHS is
\begin{equation}\label{eq:rkhs}
f_{t+1} = f_t - \gamma_t(( f_t(x_t)-y_t) K_{x_t} +\lambda f_t ), \ \ \ \ \ \mbox{for some $f_1 \in \H_K$, e.g. $f_1=0$},
\end{equation}
where
\begin{enumerate}
\item for each $t\in \N$, $(x_t,y_t)$ is drawn identically and independently according to $\rho$,
\item the regularization parameter $\lambda\geq 0$,
\item the step size $\gamma_t>0$.
\end{enumerate}

Note that for each $f$, the map $X\times Y\to \R$ given by
$(x,y)\mapsto f(x)-y$ is a real valued random variable and
$K_x:X\to \H_K$ is a $\H_K$-valued random variable. Thus $f_{t+1}$
is a random variable with values in $\H_K$ depending on
$(z_i)_{i=1}^t$. Moreover we see that $f_{t+1}\in
\span\{f_1,K_{x_i}:1\leq i\leq t\}$, a finite dimensional subspace
of $\H_K$. The derivation of (\ref{eq:rkhs}) is given in the next
section from a stochastic gradient algorithm in general Hilbert
spaces.

\bigskip

In the sequel we assume that
\begin{equation} \label{eq:ck}
C_K:=\sup_{x\in X} \sqrt{K(x,x)} < \infty.
\end{equation}
For example, the following typical kernels have $C_K=1$.

\begin{enumerate}
\item Gaussian kernel: $K:\R^n\times \R^n \to \R$ such that $K(x,x')=e^{-\|x-x'\|^2/c^2}$.
\item Homogeneous polynomial kernel: $K:R^n \times R^n \to \R$ such that $K(x,x')=\<x,x'\>^d$. By the scaling property, we can restrict $K$ to
the sphere $S^{n-1}\times S^{n-1}$.
\item Translation invariant kernels: any $K:X\times X \to \R$ such that $K(x,x')=K(x-x')$ and $K(0)=1$.
\end{enumerate}

\bigskip

In the sequel, we decompose $\|f_t - f_\rho \|_\rho$ into several
parts and give upper bounds for each of them.
Before that, we
introduce several important quantities.

First consider the minimization of the regularized least square
problem in $\H_K$,
\[ \min_{f\in \H_K} \int_Z (f(x)-y)^2 d \rho + \lambda \|f\|^2_K,  \ \ \ \ \ \lambda>0, \]
The existence and uniqueness of a minimizer is guaranteed by
(Proposition 7 in Chapter III, by Cucker and Smale
\cite{CucSma02}) which exhibits it as
\begin{equation}
f^\ast_\lambda = (L_K + \lambda I )^{-1} L_K f_\rho,
\end{equation}
where $f_\rho\in \L^2_\rho(X)$ is the regression function.
In fact, $f^\ast_\lambda$ defined in this way is also the equilibrium of the averaged
update equation of (\ref{eq:rkhs})
\begin{equation} \label{eq:avgrkhs}
\E [f_{t+1}] = \E[f_t] - \gamma_t(\E[( f_t(x_t)-y_t) K_{x_t}+\lambda f_t ]),
\end{equation}
In other words, $f^\ast_\lambda$ satisfies
\begin{equation} \label{eq:rkhs-equilibrium}
\E[( f^\ast_\lambda(x)-y) K_{x}+ \lambda f^\ast_\lambda]=0.
\end{equation}
To see this, it is enough to notice that by $L_K(f)(x)=\int_X K(x,t)f(t)d \rho_X$, we have
\[ L_K(f^\ast_\lambda)=\E_x[f^\ast_\lambda(x)K_x], \]
and
\[ L_K(f_\rho)=\E_x [[\E_{y|x}y] K_x], \]
whence the equation (\ref{eq:rkhs-equilibrium}) turns out to be $
L_K(f^\ast_\lambda) + \lambda f^\ast_\lambda = L_K( f_\rho)$,
which leads to the definition of $f^\ast_\lambda$ in (3).

Notice that the map $(x,y)\mapsto (f^\ast_\lambda(x)-y)K_x + \lambda f^\ast_\lambda$ is a $\H_K$-valued
random variable, with zero mean. Thus the following variance
\begin{equation} \label{eq:var}
\sigma^2=\E_z [ \|( f^\ast_\lambda(x)-y) K_{x} +\lambda f^\ast_\lambda\|_K^2 ],
\end{equation}
characterizes the fluctuation about the equilibrium caused by the
random sample $z=(x,y)$. If $\sigma^2=0$, we have the
deterministic gradient method (see Section 2). If $M_\rho>0$ is a
constant such that $\supp (\rho) \subseteq X\times
[-M_\rho,M_\rho]$, then Proposition \ref{prop:mnbounds} in the
next section implies
\[ \sigma^2 \leq \left(\frac{2C_K M_\rho (\lambda+C^2_K)}{\lambda} \right)^2. \]


The main purpose in this paper is to obtain a probabilistic upper bound for
\[ \| f_t - f_\rho \|_{\rho}. \]
By the triangle inequality we may write
\begin{equation} \label{eq:total}
\| f_t - f_\rho \|_{\rho} \leq  \| f_t - f^\ast_\lambda \|_{\rho} + \| f^\ast_\lambda - f_\rho \|_{\rho}.
\end{equation}
The second part of the right hand side in (\ref{eq:total}),
$\|f^\ast_\lambda- f_\rho\|_{\rho}$, is called as the
\emph{approximation error}. An upper bound for the approximation error will be given in the
end of this section. In the following we will give a probabilistic
upper bound on $\|f_t - f^\ast_\lambda\|_K$, whence via $\|f_t - f^\ast_\lambda\|_\rho \leq C_K \|f_t - f^\ast_\lambda\|_K$ we obtain
an upper bound on the sample error. Before the statement of the theorem, we define
\begin{equation}
\alpha=\frac{\lambda}{\lambda+C^2_K},
\end{equation}
whose meaning as the inverse condition number, will be discussed
in the next section.

%%%%%%%%%%%%%%%%%%%%%%%%%
%                       %
%     Theorem A         %
%                       %
%%%%%%%%%%%%%%%%%%%%%%%%%

\begin{thma}\label{thm:rkhs} Let $\DS \gamma_t = \frac{1}{(\lambda+C_K^2)t^\theta}$ ($t\in \N$) for some $\theta \in (1/2,1)$.
Then for each $t\in \N$, we may write
\begin{equation} \label{eq:intsam}
\|f_t - f^\ast_\lambda\|_K \leq \Err_{init}(t) + \Err_{samp}(t),
\end{equation}
where
\[ \Err_{init}(t) \leq e^{\frac{\alpha}{1-\theta}(1-t^{1-\theta})} \|f_1 - f^\ast_\lambda\|_K; \]
and with probability at least $1-\delta$ ($\delta \in (0,1)$) in the space $Z^{t-1}$,
\[ \Err_{samp}^2(t)\leq \frac{C_\theta \sigma^2 }{\delta(\lambda+C^2_K)^2}
 \left(\frac{1}{\alpha} \right)^{\frac{\theta}{1-\theta}}
\left(\frac{1}{t}\right)^{\theta}. \]
Here $\sigma^2$ is the variance in (\ref{eq:var}) and the positive constant $C_\theta$ satisfies
\[ C_\theta = 8 + \frac{2}{2\theta-1} \left(\frac{\theta}{e(2-2^\theta)}
\right)^{\frac{\theta}{1-\theta}}  . \]
\end{thma}

The proof will be deferred to later sections.

\begin{rem}
\label{rem:comparison}
Assume $\lambda\leq 1$ and consider the upper bound $\sigma^2 \leq \left(\frac{2C_K M_\rho (\lambda+C^2_K)}{\lambda} \right)^2$.
Then the following
holds with probability at least $1-\delta$ ($\delta\in (0,1)$),
\begin{equation} \label{eq:comparison}
\|f_t - f^\ast_\lambda\|_K \leq e^{C_1 \lambda (1-t^{1-\theta})} \|f_1 - f^\ast_\lambda\|_K
+ \frac{C_2}{\sqrt{\delta} } \left(\frac{1}{\lambda} \right)^{\frac{2-\theta}{2(1-\theta)}}
\left(\frac{1}{t}\right)^{\frac{\theta}{2}},
\end{equation}
where
\[ C_1=\frac{1}{(1-\theta)(1+C^2_K)}\ \ \ \mbox{and} \ \ \ C_2 = 2 C_K M_\rho \sqrt{C_\theta} \left( 1+C^2_K \right)^{\frac{\theta}{2(1-\theta)}}. \]
\end{rem}

\begin{rem}
\label{rem:total} In the decomposition (\ref{eq:intsam}) in
Theorem A, $\Err_{init}(t)$ has a deterministic bound and
characterizes the accumulated effect from the initial choice,
which is called as the \emph{initial error}. $\Err_{samp}(t)$
depends on the random sample and thus has a probabilistic bound,
which is called as the \emph{sample error}. We can also give upper
bounds on the \emph{approximation error},
$\|f^\ast_\lambda-f_\rho\|_\rho$.

The approximation error can be bounded if we put some regularity
assumptions on the regression function $f_\rho$. For example, the
following result appears in (Theorem 4, by Smale and Zhou
\cite{SmaZho-ShannonII}).

\begin{thm}
1) Suppose $L^{-r}_K f_\rho \in L^2_\rho (X)$ for some $r\in (0,1]$. Then
\[ \|f^\ast_\lambda-f_\rho \|_\rho \leq \lambda^{r} \| L^{-r}_K f_\rho \|_\rho. \]
2) Suppose $L^{-r}_K f_\rho \in L^2_\rho(X)$ for some $r\in (1/2,1]$.
\[ \|f^\ast_\lambda-f_\rho \|_K \leq \lambda^{r-\frac{1}{2}} \| L^{-r}_K f_\rho \|_\rho. \]
\end{thm}
Notice that since $L^{-1/2}_K$ is an isomorphism, $\H_K\to
L^2_\rho(X)$, the second condition implies $f_\rho \in \H_K$.
\end{rem}




%%%%%%%%%%%%%%%%%%%%%%%%%%%%%%%%%%%%%%%%%%%%%%%%%%%%%%%%%%%%%%%%%%%%%%%%%%%%%%%%%%%%%%%%%%%%

\section{A Stochastic Gradient Algorithm in Hilbert Spaces}

%%%%%%%%%%%%%%%%%%%%%%%%%%%%%%%%%%%%%%%%%%%%%%%%%%%%%%%%%%%%%%%%%%%%%%%%%%%%%%%%%%%%%%%%%%%%

In this section, we extend the setting in the first section to general Hilbert spaces.
Let $W$ be a Hilbert space with inner product $\<\ ,\ \>$. Consider the quadratic potential map $V: W\to \R$
given by
\begin{equation}
V(w) = \frac{1}{2} \langle A w, w \rangle + \<B,w\> + C
\end{equation}
where $A:W\to W$ is a positive definite bounded linear operator whose inverse is bounded, i.e. $\|A^{-1}\|<\infty$,
$B\in W$ and $C\in \R$. Then the gradient $\grad V: W\to W$ is given by
\[ \grad V(w)= Aw+B.  \]
$V$ has a unique minimal point $w^\ast\in W$ such that $\grad V(w^\ast)=Aw^\ast+B=0$, i.e.
\[ w^\ast = -A^{-1} B. \]
Our concern is to find an approximation of this point, when $A$, $B$ and $C$ are random variables
on a space $Z$. We give a sample complexity
analysis (i.e. the sample size sufficient to achieve an approximate minimizer with high probability)
of the so-called \emph{stochastic gradient method} given by the update formula
\begin{equation} \label{eq:stocgrad}
w_{t+1}= w_t - \gamma_t \grad V(w_t), \ \ \ \ \ \mbox{for $t=1,2,3,\ldots$}
\end{equation}
with $\gamma_t$ a positive step size. For each example $z$, the stochastic gradient of
$V_z$, $\grad V_z:W\to W$ is given by the affine map $\grad V_z(w) = A(z)
w + B(z)$, with $A(z), B(z)$ denoting the values of random
variables $A,B$ at $z\in Z$. Our analysis will benefit from this
affine structure and independent sampling. Thus (\ref{eq:stocgrad}) becomes:

\vspace{1pc}

For $t=1,2,3,\dots$, let $z_{t}$ be a sample sequence and define an update by
\begin{equation}\label{eq:wt}
w_{t+1} = w_t - \gamma_t (A_t w_t + B_t), \ \ \ \ \ \mbox{for some $w_1\in W$}
\end{equation}
where \\
1) $z_{t}\in Z$ ($t\in \N$) are drawn independently and identically according to $\rho$; \\
2) the step size $\gamma_t>0$; \\
3) the map $A: Z\to SL(W)$ is a random variable depending on $z$ with values in $SL(W)$, the vector space of symmetric bounded linear operators on $W$, and $B:Z\to W$ is a $W$-valued random variable
depending on $z$. For each $t\in \N$, let $A_t=A(z_t)$ and $B_t=B(z_t)$. \\

\medskip

From the stochastic gradient method in equation (\ref{eq:stocgrad}), we derive the equation (\ref{eq:rkhs}) for our online
algorithm in Reproducing Kernel Hilbert Spaces. Consider the Hilbert space $W=\H_K$. For fixed $z=(x,y)\in Z$, take the
following quadratic potential map
$V:\H_K\to \R$ defined by
\[ V_z(f) = \frac{1}{2}\left\{ (f(x)-y)^2 + \lambda \|f\|^2_K \right\}. \]
Recall that the gradient of $V_z$ is a map $\grad V_z:\H_K\to \H_K$ such that for all $g\in \H_K$,
\[ \< \grad V_z(f), g\>_K = D V_z(f)(g) \]
where the Fr\`{e}chet derivative at $f$, $D V_z(f):\H_K\to \R$ is the linear functional such that for $g\in \H_K$,
\[ \lim_{\|g\|\to 0} \frac{|V_z(f + g) - V_z(f) - D V_z(f) (g)|}{\|g\|} = 0. \]
Hence
\[ D V_z(f) (g) = (f(x)-y) g(x)  + \lambda \< f , g\>_K = \<(f(x) - y) K_x + \lambda f , g\>_K, \]
where the last step is due to the reproducing property $g(x)=\<g, K_x\>_K$. This gives the following proposition.

\begin{prop}
$\DS \grad V_z(f) = (f(x) - y) K_x + \lambda f$.
\end{prop}

Taking $f=f_t$ and $(x,y)=(x_t,y_t)$, by $f_{t+1}=f_t - \gamma_t \grad V_{z_t}(f_t)$, we have
\[ f_{t+1} = f_t - \gamma_t ((f_t(x_t)-y_t) K_{x_t} + \lambda f_t ),  \]
which establishes the equation (\ref{eq:rkhs}).

\medskip

In the sequel we assume that

\noindent {\bf Finiteness Condition.}\\
\noindent 1) For almost all $z\in Z$, $\amin I\leq  A(z) \leq \amax I$ ($0< \amin\leq \amax < \infty$); \\
2) $\|B(z)\|\leq \beta<\infty$ for almost all $z\in Z$.

Consider the following averaging of the equation (\ref{eq:wt}) by taking the expectation over the truncated history $(z_i)_{i=1}^{t}$,
\begin{equation} \label{eq:average}
\E_{z_1,\ldots,z_{t}}[w_{t+1}] = \E_{z_1,\ldots,z_{t-1}}[w_t] - \gamma_t ( \E_{z_t}[A_t] w_t + \E_{z_t}[B_t])
\end{equation}
where $w_t$ depends on the truncated sample up to time $t-1$, $(z_i)_{i=1}^{t-1}$. Then the equilibrium for this averaged equation (\ref{eq:average})
will satisfy
\begin{equation} \label{eq:wstar}
\E_{z_t} [A_t] w_t + \E_{z_t}[B_t] = 0 \Leftrightarrow w_t=-\E_{z_t}[A_t]^{-1} \E_{z_t}[B_t],
\end{equation}
This motivates the following definitions.


\begin{defa} 1) The equilibrium $w^\ast= -\hat{A}^{-1} \hat{B}$ where $\hat{A}= \E_z[A(z)]$ and $\hat{B}=\E_z[B(z)]$. \\
2) The inverse condition number for the family $\{A(z):z\in Z\}$, $\alpha=\amin/\amax\in (0, 1]$.
\end{defa}

For each $w\in W$, the stochastic gradient at $w$ as a map $\grad V_z(w): Z\to W$ such that $z\mapsto A(z)w+B(z)$, is a
$W$-valued random variable depending on $z$.
In particular, $\grad V_z(w^\ast)$ has zero mean, with variance defined by
$$\sigma^2 = \E[\|\grad V_z(w^\ast)\|^2]= \E_z [\| A_z w^\ast + B_z \|^2],$$
which reflects the fluctuations of $\grad V_z(w^\ast)$ caused by
the randomness of sample $z$. Observe that when $\sigma^2=0$, we
have the following deterministic gradient algorithm to minimize
$V$,
\[w_{t+1}= w_t - \gamma_t \grad V(w_t) \]
where $\grad V(w)= \hat{A}w+\hat{B}$.

Now we are ready to state the general version of the main theorem for Hilbert spaces. Here we consider
the product probability measure on $Z^{t-1}$, which makes sense since $z_i$ ($1\leq i \leq t-1$) are i.i.d. random variables. As in
the first
section, we will decompose
and give a deterministic bound on $\Err_{init}$ and a probabilistic bound on $\Err_{samp}$, respectively.

%%%%%%%%%%%%%%%%%%%%%%%%%
%           %
%     Theorem B     %
%           %
%%%%%%%%%%%%%%%%%%%%%%%%%

\begin{thmb}\label{thm:markov} Assume (\ref{eq:wt}) and the finiteness condition. Let $\gamma_t=1/\amax t^\theta$ ($t\in \N$)
for some $\theta\in(1/2,1)$. Then for each $t\in \N$, we have
\begin{equation}
\|w_t - w^\ast\| \leq \Err_{init}(t) + \Err_{samp}(t)
\end{equation}
where
\[ \Err_{init}(t)\leq e^{\frac{\alpha}{1-\theta}(1-t^{1-\theta})} \|w_1-w^\ast\|,\]
and with probability at least $1-\delta$ ($\delta\in (0,1)$)
\[ \Err_{samp}^2(t) \leq \frac{\sigma^2}{\amax^2 \delta} \psi_\theta(t,\alpha). \]
Here
\[ \psi_\theta(t,\alpha) = \sum_{k=1}^{t-1} \frac{1}{k^{2\theta}} \prod_{i=k+1}^{t-1} \left(1-\frac{\alpha}{i^\theta}\right)^2 .
\]
\end{thmb}

\begin{rem}
As in the first section, $\Err_{init}(t)$ has a deterministic upper bound and characterizes the accumulated effect from the initial choice, which is called
as the \emph{initial error}, $\Err_{samp}(t)$ depends on the random sample and thus has a probabilistic bound, which is called as the \emph{sample error}.
\end{rem}

\begin{rem}
In summary, $w_t$ in equation (\ref{eq:wt}) satisfies that for arbitrary integer $t\in \N$, the following holds with probability at least $1-\delta$ in the
space of all samples of length $t-1$, i.e. $Z^{t-1}$.
\begin{eqnarray*}
\|w_t-w^\ast\| & \leq &
e^{\frac{\alpha}{1-\theta}(1-t^{1-\theta})} \|w_1-w^\ast\| + \frac{\sqrt{\sigma^2}}{\amax \sqrt{\delta}} \psi^{1/2}_\theta(t,\alpha).
\end{eqnarray*}
When $\sigma^2=0$, we have the following convergence rate for the deterministic gradient algorithm
\[ \|w_t - w^\ast \| \leq e^{\frac{\alpha}{1-\theta}(1-t^{1-\theta})} \|w_1-w^\ast\|, \]
which is faster than any polynomial rate.
\end{rem}

\begin{prop} \label{prop:mnbounds} Let $\alpha\in(0,1]$ and $\theta\in (1/2,1)$. The following upper bounds hold for all $t\in \N$. \\
(1) $\DS \sigma^2 \leq (2\beta /\alpha)^2$; \\
(2) $\DS \psi_\theta(t,\alpha) \leq  C_\theta \left(\frac{1}{\alpha} \right)^{\frac{\theta}{1-\theta}}
\left(\frac{1}{t}\right)^{\theta}$, where
\[ C_\theta = 8 + \frac{2}{2\theta-1} \left(\frac{\theta}{e(2-2^\theta)}
\right)^{\frac{\theta}{1-\theta}}  . \]
\end{prop}

\begin{rem}
In the setting of equation (\ref{eq:rkhs}) in reproducing kernel Hilbert space, we have
\[ \beta = C_K M_\rho \ \ \ \ \ \mbox{and} \ \ \ \ \ \alpha=\frac{\lambda}{\lambda+C^2_K}, \]
whence
\[ \sigma^2 \leq \left(\frac{2C_K M_\rho(\lambda+C^2_K )}{\lambda}\right)^2.\]
\end{rem}

\begin{rem} Choose the initialization $w_1=0$ for simplicity.
Notice that $\|w^\ast\|=\|\hat{A}^{-1}\hat{B}\|\leq \beta /\amin$. Then we have the following bound with probability at least $1-\delta$,
\[ \|w_t - w^\ast \| \leq \frac{\beta}{\amin}\left(\frac{1}{t}\right)^{\frac{\theta}{2}} \left(
t^{\theta/2} e^{\frac{\alpha}{1-\theta}(1-t^{1-\theta})}  +
2\sqrt{\frac{C_\theta}{\delta}}\right). \]
\end{rem}

\begin{rem} \label{rem:Hilbert-theta1}
Consider the case that $\theta=1$ and $\alpha\in (0,1/2)$. Then by Lemma \ref{lem:yyf}-3, we obtain that
\[ \Err_{init}(t) \leq t^{-\alpha} \|w_1 - w^\ast\| \]
and
\[ \Err_{samp}(t) \leq \frac{\sqrt{\sigma^2}}{\amax \sqrt{\delta}} \psi^{1/2}_1(t,\alpha) \leq
\frac{4\beta}{\amin\sqrt{\delta(1-2\alpha)}}t^{-\alpha} . \]
Choosing $w_1=0$ and using $\|w^\ast\|\leq \beta/\amin$, we obtain that
\[ \|w_t - w^\ast \| \leq \frac{\beta}{\amin} \left( \frac{1}{t} \right)^{\alpha}
\left(1+\frac{4}{\sqrt{\delta(1-2\alpha)}}\right). \]
\end{rem}

The proof of Theorem B and Proposition \ref{prop:mnbounds} will be given in Section \ref{sec:pfs}.
Here is the proof of Theorem A from Theorem B.

\begin{proof}[Proof of Theorem A] In this case $W=\H_K$. Before applying Theorem B,
we need to rewrite the equation (\ref{eq:rkhs}) by the
notations used in Theorem B.

For any $f\in \H_K$, let the evaluation functional at $x\in X$ be $E_x:\H_K\to \R$ such that $E_x(f)=f(x)$ ($\forall x\in X$).
Denote by $E^\ast_x:\R\to \H_K$ the
adjoint operator of $E_x$ such that $\<E_x(f),y\>_{\R} = \<f,E^\ast_x(y)\>_K$ ($y\in \R$).
From this definition, we see that $E^\ast_x(y)=y K_x$.

Define the linear operator $A_x: \H_K\to \H_K$ by $A_x=E^\ast_x E_x + \lambda I$, i.e. $A_x(f)=f(x)K_x + \lambda f$, whence $A_x$ is a random variable
depending on $x$. Taking the expectation of $A_x$, we have $\hat{A}=\E_x[A_x]=L_K + \lambda I$.

Moreover, define $B_z= E^\ast_x(-y)= - y K_x \in \H_K$, which is a random variable
depending on $z=(x,y)$. Notice that the expectation of $B_z$, $\hat{B}=\E_z[B_z]=\E_x [ \E_y[-y]K_x] = -L_K f_\rho$.
For simplicity below we denote $A_t=A_{x_t}$ and $B_t=B_{z_t}$.

With these notations, the equation (\ref{eq:rkhs}) can be rewritten as
\[ f_{t+1}=f_t - \gamma_t ( A_t f_t + B_t). \]
Clearly $f^\ast_\lambda=(L_K+\lambda I)^{-1} L_K f_\rho$ satisfies $0 = \E_z[A(z) f^\ast_\lambda + B(z)] =\hat{A} f^\ast_\lambda +\hat{B}$.
Thus $f^\ast_\lambda$ is the equilibrium of the averaged equation (\ref{eq:avgrkhs}).

Notice that the positive operator $L_K$ satisfies $\|L_K\|=\sup_{x\in X} K(x,x)=C_K^2$.
Therefore $\amax= \lambda + C_K^2$, $\amin=\lambda$, and $\beta= C_K M_{\rho}$.

Finally by identifying $w_t=f_t$ and $w^\ast=f^\ast_\lambda$, the upper bound on the initial error $\Err_{init}(t)$ follows from Theorem B
and the upper bound on the sample error $\Err_{samp}(t)$ follows from Theorem B and Proposition \ref{prop:mnbounds}-2.
\end{proof}

\begin{rem} \label{rem:RKHS-theta1}
If $\theta=1$ and $\lambda<C^2_K$ (whence $\alpha\in (0,1/2)$), by Remark \ref{rem:Hilbert-theta1}, we have
\[ \|f_t - f^\ast_\lambda \|_K \leq  \left( \frac{1}{t} \right)^{\alpha}
\left(\|f^\ast_\lambda\|_K + \frac{\sqrt{\sigma^2}}{\sqrt{\delta}(\lambda+C^2_K)} \psi^{1/2}_1(t,\alpha)\right). \]
By Lemma \ref{lem:yyf}-3, we have an upper bound for $\psi_1(t,\alpha)$,
\[ \psi_1(t,\alpha) \leq  \frac{4}{1-2\alpha} t^{-2\alpha}. \]
With this upper bound and $\sigma^2 \leq (2 \beta/\alpha)^2 = 4 C_K^2 M_\rho^2 (\lambda+ C^2_K)^2/\lambda^2$, we obtain that
\[ \|f_t - f^\ast_\lambda \|_K \leq \left( \frac{1}{t} \right)^{\alpha}
\left(\|f^\ast_\lambda \|_K+ \frac{4C_K M_\rho }{\lambda\sqrt{\delta(1-2\alpha)}}\right), \]
which holds with probability at least $1-\delta$.
Notice that this upper bound has a polynomial decay $O(t^{-\alpha})$.
\end{rem}

%%%%%%%%%%%%%%%%%%%%%%%%%%%%%%%%%%%%%%%%%%%%%%%%%%%%%%%%%%%%%%%%%%%%%%%%%%%%%%%%%%%%%%%%%%%%

\section{Proof of Theorem B} \label{sec:pfs}

%%%%%%%%%%%%%%%%%%%%%%%%%%%%%%%%%%%%%%%%%%%%%%%%%%%%%%%%%%%%%%%%%%%%%%%%%%%%%%%%%%%%%%%%%%%%

In this section we shall use $\E_{z}[\cdot]$ to denote the expectation with respect to $z$. When the underlying random variable
in expectation is clear from the context, we will simply write $\E[\cdot]$.

Define the remainder vector at time $t$, $r_t = w_t - w^\ast$, which is a
random variable depending on $(z_i)_{i=1}^{t-1}\in Z^{t-1}$ when $t\geq 2$.
The following lemma gives a formula to compute $r_{t+1}$.

\begin{lem}\label{lem:rt}
\[ r_{t+1} = \prod_{i=1}^t (I-\gamma_i A_i) r_1 - \sum_{k=1}^{t}\gamma_k  \left(\prod_{i=k+1}^t (I-\gamma_i A_i) \right)(A_k w^\ast + B_k). \]
\end{lem}
\begin{proof} Since $w_{t+1}=w_t + \gamma_t (A_t w_t + B_t)$, then
\begin{eqnarray*}
r_{t+1} & = & w_{t+1}- w^\ast \\
& = & w_t - \gamma_t (A_t w_t + B_t) - (I-\gamma_t A_t) w^\ast - \gamma_t A_t w^\ast \\
& = & (I-\gamma_t A_t) r_t - \gamma_t (A_t w^\ast + B_t).
\end{eqnarray*}
The result then follows from induction on $t\in \N$.
\end{proof}

For simplicity we introduce the following notations, a symmetric linear operator $X_{k+1}^t: W\to W$ which
depends on $z_{k+1},\ldots,z_t$,
\[ X_{k+1}^t (z_{k+1},\ldots,z_t) = \prod_{i=k+1}^t (I-\gamma_i A_i), \ \ \ \ \ \ \mbox{($X_{k+1}^t=I$ if $k\geq t$)},  \]
and a vector $Y_k\in W$ which depends on $z_k$ only,
\[ Y_k(z_k) = A_k w^\ast + B_k. \]
Clearly $\E[Y_k]=0$ and $\E[\|Y_k\|^2]=\sigma^2$ for every $1\leq k\leq t$.
With this notation Lemma \ref{lem:rt} can be written as
\begin{equation} \label{eq:easy-rt}
r_{t+1}=X_1^t r_1 - \sum_{k=1}^{t} \gamma_k X_{k+1}^t Y_k,
\end{equation}
where the first term $X_1^t r_1$ reflects the accumulated error caused by the initial choice;
the second term $\sum_{k=1}^{t-1}\gamma_k X_{k+1}^tY_k$ is of zero mean and reflects the fluctuation
caused by the random sample. Based on this observation we define the \emph{initial error}
\begin{equation} \label{eq:init}
 \Err_{init}(t+1) = \| X_1^t r_1 \|
\end{equation}
and the \emph{sample error}
\begin{equation} \label{eq:samp}
\Err_{samp}(t+1) = \left\| \sum_{k=1}^{t} \gamma_k X_{k+1}^t Y_k \right\|.
\end{equation}

The main concern in this section is to obtain upper bounds on the initial error and the sample error. The following estimates are crucial in
the proofs of Theorem B and Proposition \ref{prop:mnbounds}.

\begin{prop} \label{prop:main} Let $\gamma_t = 1/\amax t^\theta$ for some $\theta\in(1/2,1]$.
For all $\alpha=\amin/\amax \in(0,1]$, the following holds.

\noindent (1) Let $\alpha'=\alpha/(1-\theta)$. Then
\[ \|X_1^t r_1 \| \leq
\left\{
\begin{array}{ll}
\DS e^{\alpha'(1-(t+1)^{1-\theta})} \| r_1 \|, & \theta\in(1/2,1); \\
\DS (t+1)^{-\alpha} \|r_1\|, & \theta=1.
\end{array}
\right.
\]
\noindent (2) $\|Y_k \| \leq 2\beta/\alpha$;

\noindent (3) $\DS \E\left[\left\|\sum_{k=1}^{t}\gamma_k X_{k+1}^tY_k\right\|^2\right] \leq
\frac{\sigma^2}{\amax^2} \psi_\theta(t+1,\alpha). $
\end{prop}

From this proposition and the following Markov's inequality, we give the proof of Theorem B.

\begin{lem} {\bf (Markov's Inequality)} Let $X$ be a nonnegative random variable.
Then for any real number $\epsilon>0$, we have
\[ \Prob\{ X\geq \epsilon \} \leq \frac{\E[X]}{\epsilon}. \]
\end{lem}

\begin{proof}[Proof of Theorem B]
By (\ref{eq:init}) and the estimation (1) in Proposition \ref{prop:main} where $\theta\in(1/2,1)$, we have
\[ \Err_{init}(t) \leq e^{\frac{\alpha}{1-\theta}(1- t^{1-\theta})} \| w_1 - w^\ast \|. \]
By (\ref{eq:samp}) and the estimation (3) in Proposition \ref{prop:main} and Markov's inequality with $X=\Err_{samp}^2(t)$, we obtain for $t\geq 2$,
\[ \Prob \{ \Err_{samp}^2(t) \leq \epsilon^2 \} \leq \frac{\sigma^2}{\epsilon^2\amax^2}\psi_\theta(t,\alpha). \]
Setting the right hand side to be $\delta\in (0,1)$, we get the probabilistic upper bound on the \emph{sample error}. It remains to check that when
$t=1$, $\Err_{init}(t)=\|w_1 - w^\ast\|$ and $\Err_{samp}(t)=0$, whence the bound still holds.
\end{proof}

Next we give the proof of Proposition \ref{prop:main}.

\begin{proof}[Proof of Proposition \ref{prop:main}]


(1) By $\amin I\leq A \leq \amax I$ and $\gamma_t=1/\amax t^\theta$ ($\theta\in (1/2,1]$), then
\begin{equation}\label{eq:Xk}
 \|X_{k+1}^t r_1\| \leq \prod_{i=k+1}^t \| I - \gamma_i A_i \| \|r_1\| \leq \prod_{i=k+1}^t \left( 1 - \frac{\alpha}{i^\theta} \right)\|r_1\|, \ \ \ \ \ \alpha=\amin/\amax;
\end{equation}
Setting $k=0$ and by (1) in Lemma \ref{lem:yyf}, we obtain the result.

(2) Note that $\DS \|w^\ast\|\leq \beta/\amin$. Thus we have
$$\|Y_k\|=\|A_k w^\ast + B_k\| \leq \|A_k\|\|w^\ast\|+\|B_k\| \leq \amax \beta/\amin + \beta= \beta(\alpha^{-1} +1)\leq 2\beta/\alpha,$$
since $\alpha\in(0,1]$. This gives part 2.

(3) Note that
\begin{eqnarray*}
\E [\|\sum_{k=1}^{t} \gamma_k X_{k+1}^tY_k \|^2] & = & \E\< \sum_{k=1}^{t} \gamma_k X_{k+1}^t Y_k, \sum_{k=1}^{t} \gamma_k X_{k+1}^t Y_k \>,  \\
& = &  \sum_{k, l =1}^{t} \gamma_k \gamma_l \E \< X_{k+1}^tY_k, X_{l+1}^tY_l \>, \\
\end{eqnarray*}
where if $k\neq l$, say $k<l$,
\begin{eqnarray*}
\gamma_k \gamma_l \E_{z_k,\ldots,z_{t}} \< X_{k+1}^tY_k , X_{l+1}^t Y_l \> = \gamma_k \gamma_l
\E_{z_{k+1},\ldots,z_{t}}[\E_{z_k|z_{k+1},\ldots,z_t} [Y_k]^T X_{k+1}^t  X_{l+1}^tY_l] = 0,
\end{eqnarray*}
by $\E[Y_k]=0$. Thus we have
\begin{eqnarray*}
\sum_{k, l =1}^{t} \gamma_k \gamma_l \E \< X_{k+1}^tY_k, X_{l+1}^tY_l \> & = & \sum_{k=1}^{t} \gamma_k^2 \E \< X_{k+1}^tY_k, X_{k+1}^tY_k \>  \leq
\sum_{k=1}^{t} \gamma_k^2 \E[\|X_{k+1}^t\|^2  \|Y_k \|^2] \\
& \leq & \frac{\sigma^2}{\amax^2} \psi_\theta(t+1,\alpha),
\end{eqnarray*}
where the last inequality is due to $\E\|Y_k\|^2=\sigma^2$ for all
$k$ and
$$\DS \sum_{k=1}^t \gamma_k^2 \|X_{k+1}^t\|^2 \leq  \sum_{k=1}^t \frac{1}{\amax^2
k^{2\theta}}\prod_{i=k+1}^t \left( 1 - \frac{\alpha}{i^\theta}
\right)^2 = \frac{1}{\amax^2} \psi_\theta(t+1,\alpha). $$
\end{proof}

Finally we derive the upper bounds for $\sigma^2$ and $\psi(t,\alpha)$ as in Proposition \ref{prop:mnbounds}.

\begin{proof}[Proof of Proposition \ref{prop:mnbounds}]
The first upper bound follows from the estimation (2) in Proposition \ref{prop:main},
\[ \sigma^2 \leq (\| Y_k \| )^2 \leq \left(\frac{2\beta}{\alpha}\right)^2 \]
for all $1\leq k \leq t$.

For $t\geq 2$, the second upper bound is an immediate result from Lemma \ref{lem:main}; for $t=1$, note that $\psi_\theta(t,\alpha)=0$ whence
the upper bound still holds.
\end{proof}



%%%%%%%%%%%%%%%%%%%%%%%%%%%%%%%%%%%%%%%%%%%%%%%%%%%%%%%%%%%%%%%%%%%%%%%%%%%%%%%%%%%%%%%%%%%%

\section{Comparison with ``Batch Learning'' Results}

%%%%%%%%%%%%%%%%%%%%%%%%%%%%%%%%%%%%%%%%%%%%%%%%%%%%%%%%%%%%%%%%%%%%%%%%%%%%%%%%%%%%%%%%%%%%

The name, ``batch learning'' is coined for the purpose of
emphasizing the case when the sample of size $t\in \N$ is exposed
to the learner in one batch, instead of one-by-one as in ``online
learning'' in this paper. In the context of RKHS, given a sample
$\z=\{z_i:i=1,\ldots,t\}$, ``batch learning'' means solving the
\emph{regularized least square} problem (\cite{EvgPonPog99},
\cite{CucSma02})
\[ f_{\lambda,\z} = \arg\min_{f\in \H_K} \frac{1}{t}\sum_{i=1}^t (f(x_i)-y_i)^2 + \lambda \< f, f \>_K, \ \ \ \ \ \lambda>0. \]
The existence and uniqueness of $f_{\lambda,\z}$ given as in (Section 6, \cite{CucSma02}) says
\[ f_{\lambda,\z}(x)= \sum_{i=1}^t a_i K(x,x_i) \]
where $a=(a_1,\ldots,a_t)$ is the unique solution of the well-posed linear system in $\R^t$
\[ (\lambda t I + K_\z) a = y, \]
with $t\times t$ identity matrix $I$, $t\times t$ matrix $K_\z$ whose $(i,j)$ entry is $K(x_i,x_j)$ and $ y=(y_1,\ldots,y_t)\in \R^t$.

A probabilistic upper bound for
$\|f_{\lambda,\z}-f^\ast_\lambda\|_\rho$ is given by
Cucker and Smale \cite{CucSma02b}, and this has been substantially improved by
De Vito, Caponnetto and Rosasco \cite{DevCapRos04}, using also some ideas from Bousquet and Elisseeff \cite{BouEli02}.
Moreover, error bounds expressed in a different form were given by
Zhang \cite{Zhang03}. A recent result, shown in Smale and Zhou \cite{SmaZho-ShannonII}, is:

\begin{thm}
\[ \| f_{\lambda,\z} - f^\ast_\lambda \|_K \leq \frac{C_{\rho,K}}{\sqrt{\delta}} \left(\frac{1}{\lambda\sqrt{t}} \right), \]
where $C_{\rho,K}=C^2_K \sqrt{\sigma^2_\rho} + 3 C^2_K \|f_\rho\|_\rho$ and
\[ \sigma_\rho^2 = \int_{X\times Y} (y- f_\rho(x))^2 d \rho. \]
\end{thm}

\begin{rem}
Notice that if $\lambda\leq 1$ without loss of generality, the equation (\ref{eq:comparison}) in Remark \ref{rem:comparison} shows
the following convergence rate
\[ \|f_t - f^\ast_\lambda\|_K \leq O\left( \left(\frac{1}{\lambda} \right)^{\frac{2-\theta}{2(1-\theta)}}
\left(\frac{1}{t}\right)^{\frac{\theta}{2}} \right), \]
where $\theta\in (1/2,1)$. Since the function $\DS \tau(\theta)=\frac{2-\theta}{2(1-\theta)}=\frac{1}{2(1-\theta)}+\frac{1}{2}$, is
an increasing function of $\theta$, then $\DS \tau(\theta) \in (3/4,\infty)$ as $\theta \in (1/2,1)$.
For small $\lambda$, when $\theta$ is close to $1/2$, the upper bound is close to $O(\lambda^{-3/4} t^{-1/4})$ which is tighter in
$\lambda$ but looser in $t$ in comparison with the theorem above; on the other hand, when $\theta$
increases, the upper bound becomes tighter in $t$ but much looser in $\lambda$.
\end{rem}


%%%%%%%%%%%%%%%%%%%%%%%%%%%%%%%%%%%%%%%%%%%%%%%%%%%%%%%%%%%%%%%%%%%%%%%%%%%%%%%%%%%%%%%%%%%

\section{Adaline}

%%%%%%%%%%%%%%%%%%%%%%%%%%%%%%%%%%%%%%%%%%%%%%%%%%%%%%%%%%%%%%%%%%%%%%%%%%%%%%%%%%%%%%%%%%%%

\begin{exmp}[Adaline or Widrow-Hoff Algorithm] The Adaline or Widrow-Hoff algorithm (p. 23, \cite{CriSha00})
is a special case of the online learning algorithm (\ref{eq:rkhs}) where the step size $\gamma_t$ is a constant $\eta$, the regularization parameter
$\lambda=0$, and the reproducing kernel is the linear kernel such that $K(x,x^\prime)= \< x, x^\prime\> + 1$ for $x,x^\prime\in X=\R^n$.
To see that, define two kernels by $K_0(x,x')=\<x,x'\>$ and $K_1(x,x')=1$. Then $\H_{K}=\H_{K_0}\oplus \H_{K_1}$. Notice
that $\H_{K_0}\backsimeq \R^n$ and $\H_{K_1}\backsimeq \R$, whence $\H_K \backsimeq \R^{n+1}$. In fact, for $w\in \R^n$ and $b\in \R$,
a function in $\H_K$ can be written as $f(x)=\sum_{i=1}^n w^i x^i + b $ for $x\in X$. By the use of the Euclidean inner product in $\R^{n+1}$, we
can write $f(x)=\<(w,b),(x,1)\>$. Therefore the Adaline
update formula
\[ (w_{t+1}, b_{t+1}) = (w_t, b_t) + \eta ( \< w, x_t \> + b - y_t ) (x_t,1), \ \ \ \ \ t\in \N, \]
can be written as the following formula, by taking the Euclidean inner product of both sides with the vector $(x,1)\in \R^{n+1}$,
\[ f_{t+1} = f_t + \eta (f_t(x_t) - y_t) K_{x_t}. \]
This is equivalent to set $\gamma_t=\eta$ and $\lambda=0$ in the online learning algorithm (\ref{eq:rkhs}).
\end{exmp}

The case for fixed step size and zero regularization parameter is
not included in Theorem A or B. In the case of non-stochastic
samples, Cesa-Bianchi et al. \cite{CesLonWar96} have some worst case analysis on the upper bounds for the following quantity,
\[ \sum_{t=1}^T (\<w_t,x_t\>-y_t)^2 - \min_{\|w\|\leq W} \sum_{t=1}^T (\<w,x_t\>-y_t)^2. \]
Adam Kalai has shown us how one might convert these results of Cesa-Bianchi et al. to a form comparable to
Theorem A. Beyond the square loss function above, some related works include \cite{KivSmoWil04} which presents a general gradient descent method in RKHS for
bounded differentiable functions, and \cite{Zinkevich03} which studies the gradient method with arbitrary differentiable convex loss functions.
These works suggest different schemes on choosing the step size parameter and how these choices might affect the convergence rate under various conditions.

\vspace{2pc}


%%%%%%%%%%%%%%%%%%%%%%%%%%%%%%%%%%%%%%%%%%%%%%%%%%%%%%%%%%%%%%%%%%%%%%%%%%%%%%%%%%%%%%%%%%%%

\section*{Appendix A: Some Estimates}

%%%%%%%%%%%%%%%%%%%%%%%%%%%%%%%%%%%%%%%%%%%%%%%%%%%%%%%%%%%%%%%%%%%%%%%%%%%%%%%%%%%%%%%%%%%%

\renewcommand{\thesection}{A}
\setcounter{equation}{0}
\setcounter{thm}{0}
\renewcommand{\thethm}{A.\arabic{thm}}
\renewcommand{\theequation}{A-\arabic{equation}}

The following Lemma gives an upper bound for
\[ \psi_\theta(t,\alpha) = \sum_{k=1}^{t-1} \frac{1}{k^{2\theta}} \prod_{i=k+1}^{t-1} \left(1-\frac{\alpha}{i^{\theta}}\right)^2. \]

\begin{lem}[Main Analytic Estimate] \label{lem:main}
Let $\alpha\in (0,1]$ and $\theta\in (1/2,1)$. Then for $t\in \N$,
\begin{eqnarray*}
\psi_\theta(t+1,\alpha) & \leq &
C_\theta \left(\frac{1}{\alpha} \right)^{\frac{\theta}{1-\theta}} \left(\frac{1}{t+1}\right)^{\theta},
\end{eqnarray*}
where
\[ C_\theta = 8+\frac{2}{2\theta-1} \left(\frac{\theta}{e(2-2^\theta)}
\right)^{\frac{\theta}{1-\theta}}
 . \]
\end{lem}

\begin{proof}
The following fact will be used repeatedly in this section,
\begin{equation}\label{eq:logubd}
\ln (1+x) \leq x, \ \ \ \ \mbox{for all $x>-1$}.
\end{equation}
Thus we have
\begin{eqnarray*}
\sum_{i=k+1}^t \ln \left( 1-\frac{\alpha}{i^\theta} \right)^2 & \leq &
-2\alpha \sum_{i=k+1}^t \frac{1}{i^\theta} \leq -2\alpha \int_{k+1}^{t+1} \frac{1}{x^\theta} d x,
\end{eqnarray*}
which equals
\[ \frac{2\alpha}{1-\theta} \left((k+1)^{1-\theta}-(t+1)^{1-\theta} \right) \]
if $\theta\in (1/2,1)$.

From this estimate it follows,
\[ \psi_\theta(t+1,\alpha) \leq e^{-2\alpha'(t+1)^{1-\theta}} \sum_{k=1}^{t} \frac{1}{k^{2\theta}}
e^{2\alpha'(k+1)^{1-\theta}} = S_1 + S_2 \]
where $\DS \alpha'=\frac{\alpha}{1-\theta}$ and
\[ S_1 = e^{-2\alpha'(t+1)^{1-\theta}} \sum_{k=1}^{\lfloor\frac{t-1}{2}\rfloor} \frac{1}{k^{2\theta}} e^{2\alpha'(k+1)^{1-\theta}},\]
\[ S_2 = e^{-2\alpha'(t+1)^{1-\theta}} \sum_{k=\lfloor\frac{t+1}{2}\rfloor}^{t} \frac{1}{k^{2\theta}} e^{2\alpha'(k+1)^{1-\theta}},\]
where $\lfloor x \rfloor$ denotes the largest integer no bigger than $x$.

Next we give upper bounds on $S_1$ and $S_2$. First,
\begin{eqnarray*}
S_1 & \leq & e^{-2 \alpha' (1-2^{\theta-1}) (t+1)^{1-\theta}} \sum_{k=1}^{\lfloor\frac{t-1}{2}\rfloor} \frac{1}{k^{2\theta}}
\leq e^{-2 \alpha' (1-2^{\theta-1}) (t+1)^{1-\theta}} \int_{1/2}^{t/2} \frac{1}{x^{2\theta}} d x \\
& = &  e^{-2 \alpha' (1-2^{\theta-1}) (t+1)^{1-\theta} }\frac{1}{1-2\theta} \left(\left(\frac{t}{2}\right)^{1-2\theta}- \left(\frac{1}{2}\right)^{1-2\theta}\right)
\leq  \frac{2}{2 \theta-1} e^{-2 \alpha' (1-2^{\theta-1}) (t+1)^{1-\theta}}
\end{eqnarray*}
as $\theta\in (1/2,1)$. To give a polynomial upper bound for $\exp\{-2 \alpha' (1-2^{\theta-1}) (t+1)^{1-\theta}\}$, we use
the fact that for
any $c>0$, $a>0$, and $x\in (0,\infty)$,
\[ e^{-cx} \leq \left(\frac{a}{ec}\right)^a x^{-a}. \]
To see this, it is enough to observe that the function $f(x)=x^a / e^{cx}$ is maximized at $x=a/c$.
Let $a=(1/\theta-1)^{-1}$, $c= 2 \alpha' (1-2^{\theta-1}) $, and $x=(t+1)^{1-\theta}=(t+1)^{\theta(1/\theta-1)}$,
then,
\[ e^{-2 \alpha' (1-2^{\theta-1}) (t+1)^{1-\theta}} \leq \left(\frac{\theta}{e\alpha(2-2^\theta)}
\right)^{\frac{\theta}{1-\theta}} (t+1)^{-\theta},
\]
Thus for $\theta\in (1/2,1)$, $\alpha\in (0,1)$ and $t\in \N$,
\[ S_1 \leq \frac{2}{2\theta-1} \left(\frac{\theta}{e\alpha(2-2^\theta)}
\right)^{\frac{\theta}{1-\theta}} (t+1)^{-\theta}.\]

Second,
notice that $\DS \frac{1}{\lfloor (t+1)/2 \rfloor} \leq \frac{2}{t} \leq \frac{4}{t+1}$ (for $t\in \N$),
then let $\DS p(t)=\frac{e^{2\alpha'(t+1)^{1-\theta}}}{t^\theta}$ and we have
\begin{eqnarray*}
S_2 & \leq & e^{-2 \alpha' (t+1)^{1-\theta}} \frac{4^\theta}{(t+1)^\theta}\left( p(t)+\sum_{k=\lfloor\frac{t+1}{2}\rfloor}^{t-1}
\frac{1}{k^{\theta}} e^{2\alpha'(k+1)^{1-\theta}} \right) \\
& \leq & 2^{2\theta} e^{-2 \alpha' (t+1)^{1-\theta}} (t+1)^{-\theta}\left( p(t)+\int_{t/2-1}^{t} \frac{1}{x^{\theta}} e^{2\alpha'(x+1)^{1-\theta}} d x\right),
\ \ \ \ \mbox{for $t\geq 4$}, \\
\end{eqnarray*}
where
\begin{eqnarray*}
\int_{t/2-1}^{t} \frac{1}{x^{\theta}} e^{2\alpha'(x+1)^{1-\theta}} d x & \leq &  \int_{t/2-1}^{t} \frac{2^\theta}{(x+1)^{\theta}}
e^{2\alpha'(x+1)^{1-\theta}} d x , \ \ \ \mbox{by $\DS \frac{1}{x}\leq \frac{2}{x+1}$ for $t\geq 4$} \\
& = & \frac{2^\theta}{1-\theta} \int_{(t/2)^{1-\theta}}^{(t+1)^{1-\theta}} e^{2\alpha'y} d y , \ \ \ \ \ \ \mbox{by $\DS y=(x+1)^{1-\theta}$} \\
& = & \frac{2^{\theta-1}}{\alpha'(1-\theta)} e^{2\alpha'(t+1)^{1-\theta}} \left(1- e^{2\alpha'((t/2)^{1-\theta}-(t+1)^{1-\theta})} \right) \leq
\frac{1}{\alpha} e^{2\alpha'(t+1)^{1-\theta}},
\end{eqnarray*}
whence
\begin{eqnarray*}
S_2 & \leq & 2^{2\theta}(t+1)^{-\theta} \left( t^{-\theta} + \frac{1}{\alpha} \right) \leq \frac{8}{\alpha} (t+1)^{-\theta}, \ \ \ \ \mbox{for $t\geq 4$}.
\end{eqnarray*}
It is easy to check $\DS \psi(t+1,\alpha)\leq 2 \leq \frac{8}{\alpha} (t+1)^{-\theta}$ for $1\leq t \leq 3$.

Therefore for $t\in \N$,
\begin{eqnarray*}
\psi_\theta(t+1,\alpha) & \leq & S_1 + S_2 \leq
\left(\frac{2}{2\theta-1} \left(\frac{\theta}{e\alpha(2-2^\theta)}
\right)^{\frac{\theta}{1-\theta}} +
\frac{8}{\alpha} \right) (t+1)^{-\theta} \\
& = & \left(\frac{2}{2\theta-1} \left(\frac{\theta}{e(2-2^\theta)}
\right)^{\frac{\theta}{1-\theta}} +
8\alpha^{\frac{2\theta-1}{1-\theta}} \right) \left( \frac{1}{\alpha}\right)^{\frac{\theta}{1-\theta}}(t+1)^{-\theta} \\
& \leq & \left(\frac{2}{2\theta-1} \left(\frac{\theta}{e(2-2^\theta)}
\right)^{\frac{\theta}{1-\theta}} +
8 \right) \left( \frac{1}{\alpha}\right)^{\frac{\theta}{1-\theta}}(t+1)^{-\theta},
\end{eqnarray*}

\noindent where the last step is due to $\alpha^{\frac{2\theta-1}{1-\theta}}<1$ as $\alpha\in (0,1)$.
\end{proof}

The following lemma is also useful in the various upper bound estimations in Proposition \ref{prop:main}.

\begin{lem}\label{lem:yyf}
(1) For $\alpha\in (0,1]$ and $\theta\in [0,1]$,
\[ \prod_{i=k+1}^t \left(1-\frac{\alpha}{i^\theta}\right) \leq
\left\{
\begin{array}{ll}
\DS \exp \left(\frac{2\alpha}{1-\theta} \left((k+1)^{1-\theta}-(t+1)^{1-\theta} \right)\right), & \theta\in [0,1) \\
\DS \left(\frac{k+1}{t+1}\right)^{\alpha}, & \theta = 1
\end{array}
\right.
\]

%noindent (2) For $\alpha\in (0,1]$ and $\theta\in [0,1]$,
%\[ \frac{1}{k^\theta}\prod_{i=k+1}^t \left(1-\frac{\alpha}{i^\theta}\right) \leq
%\left\{
%\begin{array}{ll}
%\DS \frac{1}{k^\theta}\exp \left(\frac{2\alpha}{1-\theta} \left((k+1)^{1-\theta}-(t+1)^{1-\theta} \right)\right), & \theta\in [0,1) \\
%\DS \frac{2}{(t+1)^\alpha}, & \theta = 1
%\end{array}
%\right.
%\]

\noindent (2) For $\alpha\in (0,1]$ and $\theta\in [0,1]$,
\[ \sum_{k=1}^{t} \frac{1}{k^\theta} \prod_{i=k+1}^t \left(1-\frac{\alpha}{i^\theta}\right) \leq \frac{3}{\alpha}; \]

\noindent (3)
If $\theta=1$ and for $\alpha\in (0,1]$,
\begin{eqnarray*}
\psi_1(t+1,\alpha) & = & \sum_{k=1}^{t} \frac{1}{k^{2}} \prod_{i=k+1}^t \left(1-\frac{\alpha}{i}\right)^2 \\
& \leq
&
\left\{
\begin{array}{ll}
\DS \frac{4}{1-2\alpha}(t+1)^{-2\alpha}, & \alpha\in(0,1/2); \\
\DS 4 (t+1)^{-1}\ln(t+1), &\alpha=1/2; \\
\DS \frac{6}{2\alpha-1} (t+1)^{-1}, & \alpha\in(1/2,1); \\
\DS 6 (t+1)^{-1}, & \alpha=1.

\end{array}
\right.
\end{eqnarray*}
\end{lem}


\begin{proof}

(1) By the inequality (\ref{eq:logubd}), we have for $\theta \in [0,1]$,
\[ \ln \left(1 - \frac{\alpha}{i^\theta}\right) \leq \frac{-\alpha}{i^\theta}. \]
Thus
\begin{equation} \label{ineq:yyf1}
\sum_{i=k+1}^t \ln \left( 1-\frac{\alpha}{i^\theta} \right) \leq  -\alpha \sum_{i=k+1}^t \frac{1}{i^\theta}
 \leq  -\alpha \int_{k+1}^{t+1} \frac{1}{x^\theta} d x
\end{equation}
which equals
\[\frac{\alpha}{1-\theta} \left((k+1)^{1-\theta}-(t+1)^{1-\theta} \right), \]
if $\theta\in [0,1)$, and

\[ \ln \left(\frac{k+1}{t+1}\right)^\alpha, \]
if $\theta=1$. Taking the exponential gives the inequality.

(2) If $\theta\in [0,1)$, from (1) we have
\[ \frac{1}{k^\theta}\prod_{i=k+1}^t \left(1-\frac{\alpha}{i^\theta}\right) \leq e^{-\frac{2\alpha}{1-\theta}(t+1)^{1-\theta}}
\frac{1}{k^\theta}e^{\frac{2\alpha}{1-\theta} (k+1)^{1-\theta}}, \]
whence
\begin{eqnarray*}
\sum_{k=1}^{t} \frac{1}{k^\theta}\prod_{i=k+1}^t \left(1-\frac{\alpha}{i^\theta}\right) & \leq & e^{-\frac{2\alpha}{1-\theta}(t+1)^{1-\theta}}
\left( \sum_{k=1}^{t-1} \frac{1}{k^\theta}e^{\frac{2\alpha}{1-\theta} (k+1)^{1-\theta}} + \frac{1}{t^\theta} e^{\frac{2\alpha}{1-\theta} (t+1)^{1-\theta}} \right)
\end{eqnarray*}
where
\begin{eqnarray*}
\sum_{k=1}^{t-1} \frac{1}{k^\theta}e^{\frac{2\alpha}{1-\theta} (k+1)^{1-\theta}} & \leq & 2^\theta \sum_{k=1}^{t-1}
\left(\frac{1}{k+1}\right)^\theta e^{\frac{2\alpha}{1-\theta} (k+1)^{1-\theta}} \\
& \leq & 2 \int_2^{t+1}  e^{\frac{2\alpha}{1-\theta} x^{1-\theta}} x^{-\theta} d x \leq
\frac{1}{\alpha} e^{\frac{2\alpha}{1-\theta}(t+1)^{1-\theta}}.
\end{eqnarray*}
Therefore
\[ e^{-\frac{2\alpha}{1-\theta}(t+1)^{1-\theta}} \left( \sum_{k=1}^{t-1} \frac{1}{k^\theta}e^{\frac{2\alpha}{1-\theta} (k+1)^{1-\theta}} +
\frac{1}{t^\theta} e^{\frac{2\alpha}{1-\theta} (t+1)^{1-\theta}} \right) \leq \frac{1}{\alpha} + t^{-\theta} < \frac{3}{\alpha}. \]

If $\theta=1$, from the inequality (\ref{ineq:yyf1}),
\begin{eqnarray*}
\sum_{k=1}^{t-1} \frac{1}{k} \prod_{i=k+1}^t \left( 1 - \frac{\alpha}{i}
\right) & \leq & \sum_{k=1}^{t-1} \frac{1}{k}\left(\frac{k+1}{t+1}\right)^\alpha
\leq \frac{2}{t^\alpha} \sum_{k=1}^{t-1} \frac{(k+1)^\alpha}{k+1}  = \frac{2}{t^\alpha} \sum_{k=1}^{t-1} (k+1)^{\alpha-1}  \\
& \leq  & \frac{2}{t^\alpha} \int_1^{t} x^{\alpha-1} d x,
\end{eqnarray*}
where if $\alpha=1$,
\[ \frac{2}{t^\alpha} \int_1^{t} x^{\alpha-1} d x = 2; \]
and if $0<\alpha<1$,
\[ \frac{2}{t^\alpha} \int_1^{t} x^{\alpha-1} d x = \frac{2}{\alpha}\left(\frac{t^\alpha-1}{t^\alpha}\right) \leq \frac{2}{\alpha}.\]
Therefore
\[\sum_{k=1}^{t} \frac{1}{k} \prod_{i=k+1}^t \left( 1 - \frac{\alpha}{i}\right)\leq t^{-\theta} + \frac{2}{\alpha} \leq \frac{3}{\alpha}, \]
which completes the proof of part 2.

(3) If $\theta=1$,
using the inequality (\ref{eq:logubd}), we have
\[ \sum_{i=k+1}^t \ln \left( 1-\frac{\alpha}{i} \right)^2 \leq -2\alpha \sum_{i=k+1}^t \frac{1}{i}
\leq -2\alpha \int_{k+1}^{t+1} \frac{1}{x} d x =  \ln \left(\frac{k+1}{t+1}\right)^{2\alpha}. \]
Thus
\begin{eqnarray*}
\psi_1(t+1,\alpha) & \leq & t^{-2} + \sum_{k=1}^{t-1}
\frac{1}{k^2}\left(\frac{k+1}{t+1}\right)^{2\alpha}
\leq \frac{4}{(t+1)^2}+ \frac{2^{2\alpha}}{(t+1)^{2\alpha}} \sum_{k=1}^{t-1} k^{2\alpha-2} \\
& \leq & \frac{4}{(t+1)^2} + \frac{2^{2\alpha}}{(t+1)^{2\alpha}} \int_{1/2}^{t-1/2} x^{2\alpha-2} d x,
\end{eqnarray*}
where if $\alpha\in(0,1/2)$,
\[
r.h.s.  =
\frac{4}{(t+1)^2}+\frac{2^{2\alpha}}{1-2\alpha}(t+1)^{-2\alpha} \left( 2^{1-2\alpha} - (t-1/2)^{2\alpha-1} \right)
\leq \left(2+\frac{2}{1-2\alpha}\right) (t+1)^{-2\alpha}\leq \frac{4}{1-2\alpha} (t+1)^{-2\alpha};
\]
if $\alpha=1/2$,
\[
r.h.s.
= \frac{4}{(t+1)^2}+\frac{2}{t+1} (\ln(t-1/2)-\ln 1/2)
 \leq \left(\frac{2}{t+1}+\ln2\right)\frac{2}{t+1} \ln(t+1)\leq \frac{4}{t+1}\ln(t+1);
\]
if $\alpha\in(1/2,1)$,
\[
r.h.s. =  \frac{4}{(t+1)^2}+\frac{2^{2\alpha}}{2\alpha-1}(t+1)^{-2\alpha} \left( (t-1/2)^{2\alpha-1} - (1/2)^{2\alpha-1}\right)
\leq \left(\frac{4}{t+1}+\frac{4}{2\alpha-1}\right) (t+1)^{-1} \leq \frac{6}{2\alpha-1}(t+1)^{-1};
\]
and if $\alpha=1$,
\[
r.h.s. = \frac{4}{(t+1)^2}+ \frac{4}{(t+1)^2} (t-1) \leq 6 (t+1)^{-1}.
\]
This finishes the proof of the fourth part.
\end{proof}

%%%%%%%%%%%%%%%%%%%%%%%%%%%%%%%%%%%%%%%%%%%%%%%%%%%%%%%%%%%%%%%%%%%%%%%%%%%%%%%%%%%%%%%%%%%%

\section*{Appendix B: Generalized Bennett's Inequality}

%%%%%%%%%%%%%%%%%%%%%%%%%%%%%%%%%%%%%%%%%%%%%%%%%%%%%%%%%%%%%%%%%%%%%%%%%%%%%%%%%%%%%%%%%%%%

\renewcommand{\thesection}{B}
\setcounter{equation}{0}
\setcounter{thm}{0}
\renewcommand{\thethm}{B.\arabic{thm}}
\renewcommand{\theequation}{B-\arabic{equation}}


In the direction of proving an exponential version of the main theorems with $1/\delta$ replaced by $\log 1/\delta$, it has seemed useful
for us to consider Bennett's inequality for random variables in a Hilbert space. In the mean time, such a theorem was found useful in other
work to appear. Thus we include Appendix B.

The following theorem might be considered as a generalization of
Bennett's inequality for independent sums in Hilbert spaces, whose
counterpart in real random variables is given in (Theorem 3,
Smale and Zhou \cite{SmaZho-ShannonII}).
\begin{thm}[Generalized Bennett] \label{thm:GBennett}
Let $\H$ be a Hilbert space, $\xi_i\in \H$ ($i=1,\ldots,n$) be independent random variables and $T_i: \H\to \H$ be deterministic linear operators. Define
$\tau_i = \| T_i \|$ and $\tau_\infty=\sup_i \tau_i$. Suppose that for all $i$ almost surely $\|\xi_i\|\leq M<\infty$.
Define $\sigma_i^2=\E\|\xi_i\|^2$ and $\sigma^2_\tau=\sum_{i=1}^n \tau_i \sigma^2_i$. Then
\[ \P \left\{ \left\|\sum_{i=1}^n T_i (\xi_i-\E \xi_i) \right\| \geq \epsilon \right\} \leq
2 \exp \left\{-\frac{\sigma^2_\tau}{\tau_\infty M^2} g\left(\frac{M\epsilon}{\sigma^2_\tau}\right)\right\} \]
where $\DS g(t)=(1+t)\log(1+t)-t$ for all $t\geq 0$. Considering that $\DS g(t)\geq \frac{t}{2}\log(1+t)$, then
\[ \P \left\{ \left\|\sum_{i=1}^n T_i (\xi_i-\E \xi_i) \right\| \geq \epsilon \right\} \leq
2 \exp \left\{-\frac{\epsilon}{2\tau_\infty M} \log\left(1+\frac{M\epsilon}{\sigma^2_\tau}\right)\right\} \]
\end{thm}

The proof needs the following lemma due to I.F. Pinelis and A.I. Sakhanenko \cite{PinSak85}. Its current
form is taken from Theorem 3.3.4(a) in Yurinsky \cite{Yurinsky95}.
\begin{lem}[Pinelis and Sakhanenko, 1985]
Let $\xi_i\in \H$ ($i=1,\ldots,n$) be a sequence of independent
random variables with values in a Hilbert space $\H$ and
$\E[\xi_i]=0$. Then for any $t>0$,
\[ \E \left[\cosh\left(t \|\sum_{i=1}^n \xi_i \| \right) \right] \leq \prod_{j=1}^n \E \left( e^{t \|\xi_j\|} - t \|\xi_j\| \right). \]
\end{lem}

\begin{proof}[Proof of Theorem \ref{thm:GBennett}.] Without loss of generality we assume $\E[\xi_i]=0$. For arbitrary $s>0$, by Markov's inequality,
\begin{eqnarray*}
\P \left\{ \|\sum_{i=1}^n T_i \xi_i \| \geq \epsilon \right\} & =  &\P \left\{ \exp\left(s \|\sum_{i=1}^n T_i \xi_i \|\right) \geq e^{s\epsilon} \right\} \\
& \leq & e^{-s \epsilon} \E \exp\left(s \|\sum_{i=1}^n T_i \xi_i \|\right) \\
& \leq & 2 e^{-s \epsilon} \E \cosh \left(s \|\sum_{i=1}^n T_i \xi_i \|\right)
\end{eqnarray*}
where the last inequality is due to $e^x \leq e^x + e^{-x}=2\cosh(x)$. Then by Lemma B.1,
\[ \P \left\{ \|\sum_{i=1}^n T_i \xi_i \| \geq \epsilon \right\} \leq 2 e^{-s \epsilon} \prod_{j=1}^n \E \left( e^{s \|T_j \xi_j\|} - s \|T_j \xi_j\| \right).\]
Denote
\[ I = 2 e^{-s \epsilon} \prod_{j=1}^n \E \left( e^{s \|T_j \xi_j\|} - s \|T_j \xi_j\| \right). \]
For each $1\leq j\leq n$, considering $\E\|\xi_j\|^2 = \sigma_j^2$ and $\|\xi_j\|\leq M$ almost surely,
\begin{eqnarray*}
\E \left( e^{s \|T_j \xi_j\|} - s \|T_j \xi_j\| \right) & = & 1 + \sum_{k=2}^n \frac{s^k \E \|T_j \xi_j\|^k}{k\!} \\
& \leq & 1 + \sum_{k=2}^n \frac{s^k \tau_\infty^{k-1} M^{k-2}}{k\!} \tau_j \sigma^2_j \\
& \leq & \exp\left(\sum_{k=2}^n \frac{s^k \tau_\infty^{k-1} M^{k-2}}{k\!} \tau_j \sigma^2_j\right) \\
& = & \exp\left( \frac{e^{s\tau_\infty M}-1-s\tau_\infty M}{\tau_\infty M^2} \tau_j \sigma^2_j \right)
\end{eqnarray*}
where the second last inequality is due to $1 + x \leq e^x$ for all $x$. Therefore
\[ I \leq \exp\left\{-s \epsilon +  \frac{e^{s\tau_\infty M}-1-s\tau_\infty M}{\tau_\infty M^2} \sum_{j=1}^n \tau_j \sigma^2_j \right\}, \]
where the right hand side is minimized at
\[ s_0 = \frac{1}{\tau_\infty M} \log \left( 1 + \frac{M\epsilon}{\sum_{j=1}^n \tau_j \sigma^2_j } \right) .\]
Notice that $\sigma^2_\tau =\sum_{j=1}^n \tau_j \sigma^2_j$, then with this choice we arrive at
\[ I \leq \exp \left\{ -\frac{\sigma^2_\tau}{\tau_\infty M^2} g\left(\frac{M\epsilon}{\sigma^2_\tau} \right)\right\}, \]
where the function $g(t)=(1+t)\log(1+t)-t$ for all $t\geq 0$. This is the first inequality.

Moreover, we can check the lower bound of $g$,
\[ g(t) \geq \frac{t}{2} \log(1+t), \]
which leads to the second inequality.
\end{proof}


By taking $\DS T_i=\frac{1}{n} I$, the following corollary gives a
form of Bennett's inequality for random variables in Hilbert
spaces.
\begin{cor}[Bennett]
Let $\H$ be a Hilbert space and $\xi_i\in \H$ ($i=1,\ldots,n$) be independent random variables such that $\|\xi_i\|\leq M$ and
$\E\|\xi_i\|^2 \leq \sigma^2$ for all $i$.
Then
\[ \P \left\{ \left\|\frac{1}{n}\sum_{i=1}^n [\xi_i-\E \xi_i] \right\| \geq \epsilon \right\} \leq
2 \exp \left\{- \frac{n\sigma^2}{M^2} g \left(\frac{M\epsilon}{\sigma^2}\right)\right\}. \]
\end{cor}
Noticing that $\DS g(t)\geq \frac{t^2}{2(1+t/3)}$, the corollary leads to the following Bernstein's inequality for independent sums in
Hilbert spaces.
\begin{cor}[Bernstein]
Let $\H$ be a Hilbert space and $\xi_i\in \H$ ($i=1,\ldots,n$) be independent random variables such that $\|\xi_i\|\leq M$ and
$\E\|\xi_i\|^2 \leq \sigma^2$ for all $i$.
Then
\[ \P \left\{ \left\|\frac{1}{n}\sum_{i=1}^n [\xi_i-\E \xi_i] \right\| \geq \epsilon \right\} \leq
2 \exp \left\{- \frac{n\epsilon^2}{2(\sigma^2+M\epsilon/3)} \right\}. \]
\end{cor}
Yurinsky \cite{Yurinsky95} also gives Bernstein's inequalities for independent
sums in Hilbert spaces and Banach spaces. The
following result is a varied form of Theorem 3.3.4(b) in
\cite{Yurinsky95}. Note that it is weaker than the form above in that the constant $1/3$ changes to
$1$.

\begin{thm}
Let $\xi_i$ be independent random variables with values in a
Hilbert space $\H$. Suppose that for all $i$ almost surely
$\|\xi_i\|\leq M<\infty$ and $\E\|\xi_i\|^2\leq \sigma^2 <\infty$.
Then for $n\geq 0$
\[ \P \left\{ \left\|\frac{1}{n}\sum_{i=1}^n (\xi_i-\E[\xi_i]) \right\| \geq \epsilon \right\}
\leq 2 \exp \left\{- \frac{n\epsilon^2}{2(\sigma^2+ M\epsilon)}
\right\}.\]
\end{thm}


\bibliographystyle{plain}
%\bibliography{yao}

\begin{thebibliography}{10}

\bibitem{Benaim99}
M.~Bena\"{\i}m.
\newblock Dynamics of stochastic approximations.
\newblock In {\em Le Seminaire de Probabilites, Lectures Notes in Mathematics,
  Vol 1709}, pages 1--68. Springer-Verlag, 1999.

\bibitem{BerTsi96}
D.~P. Bertsekas and J.~N. Tsitsiklis.
\newblock {\em Neuro-Dynamic Programming}.
\newblock Athena Scientific, Belmont, Massachusetts, 1996.

\bibitem{BouEli02}
O.~Bousquet and A.~Elisseeff.
\newblock Stability and generalization.
\newblock {\em Journal of Machine Learning Research}, (2):499--526, 2002.

\bibitem{CesLonWar96}
N.~Cesa-Bianchi, P.~M. Long, and M.~K. Warmuth.
\newblock Worst-case quadratic loss bounds for prediction using linear
  functions and gradient descent.
\newblock {\em IEEE Transactions on Neural Networks}, 7(3):604--619, 1996.

\bibitem{CriSha00}
N.~Cristianini and J.~Shawe-Taylor.
\newblock {\em An Introduction to Support Vector Machines and Other
  Kernel-based Learning Methods}.
\newblock Cambridge Unversity Press, 2000.

\bibitem{CucSma02b}
F.~Cucker and S.~Smale.
\newblock Best choices for regularization parameters in learning theory.
\newblock {\em Foundations Comput. Math.}, 2(4):413--428, 2002.

\bibitem{CucSma02}
F.~Cucker and S.~Smale.
\newblock On the mathematical foundations of learning.
\newblock {\em Bull. of the Amer. Math. Soc.}, 29(1):1--49, 2002.

\bibitem{DevCapRos04}
E.~De~Vito, A.~Caponnetto, and L.~Rosasco.
\newblock Model selection for regularized least-squares algorithm in learning
  theory.
\newblock {\em Foundations of Computational Mathematics}, 2004.
\newblock preprint.

\bibitem{Duflo97}
M.~Duflo.
\newblock Cibles atteignables avec une probabilit\'{e} positive d'apr\'{e}s m.
  benaim.
\newblock {\em Unpublished manuscript}, 1997.

\bibitem{Duflo96}
Marie Duflo.
\newblock {\em Algorithmes Stochastiques}.
\newblock Springer-Verlag, Berlin, Heidelberg, 1996.

\bibitem{EvgPonPog99}
T.~Evgeniou, M.~Pontil, and T.~Poggio.
\newblock Regularization networks and support vector machines.
\newblock {\em Advances of Computational Mathematics}, 13(1):1--50, 1999.

\bibitem{Gyofi80}
L.~Gy\"{o}rfi.
\newblock Stochastic approximation from ergodic sample for linear regression.
\newblock {\em Z. Wahrscheinlichkeitstheorie und verwandte Gebiete}, 54:47--55,
  1980.

\bibitem{GyoKohKrzWal02}
L\'{a}szl\'{o} Gy\"{o}rfi, Michael Kohler, Adam Krzy\.{z}ak, and Harro Walk.
\newblock {\em A Distribution-Free Theory of Nonparametric Regression}.
\newblock Springer-Verlag, New York, 2002.

\bibitem{KieWol52}
J.~Kiefer and J.~Wolfowitz.
\newblock Stochastic estimation of the maximum of a regression function.
\newblock {\em The Annals of Mathematical Statistics}, 23:462--466, 1952.

\bibitem{KivSmoWil04}
J.~Kivinen, A.~J. Smola, and R.~C. Williamson.
\newblock Online learning with kernels.
\newblock {\em IEEE Transactions on Signal Processing}, 52(8):2165--2176, 2004.

\bibitem{KusYin03}
H.~J. Kushner and G.~G. Yin.
\newblock {\em Stochastic Approximations and Recursive Algorithms and
  Applications}.
\newblock Springer-Verlag, Berlin, Heidelberg, 2003.

\bibitem{PinSak85}
I.F. Pinelis and A.I. Sakhanenko.
\newblock Remarks on inequalities for probabilities of large deviations.
\newblock {\em Theory of Probability and Applications}, 30(1):143--148, 1985.

\bibitem{RobMon51}
H.~Robbins and S.~Monro.
\newblock A stochastic approximation method.
\newblock {\em The Annals of Mathematical Statistics}, 22(3):400--407, 1951.

\bibitem{RobSie71}
H.~Robbins and D.~Siegmund.
\newblock A convergence theorem for nonnegative almost supermartingales and
  some applications.
\newblock In J.~S. Rustagi, editor, {\em Optimizing Methods in Statistics},
  pages 233--257. Academic Press, New York, 1971.

\bibitem{SmaZho-ShannonII}
S.~Smale and D.-X. Zhou.
\newblock Shannon sampling ii. connections to learning theory.
\newblock {\em submitted to Appl. Comput. Harmonic Anal.}, 2005.

\bibitem{Tadic04}
V.~B. Tadic.
\newblock On the almost sure rate of convergence of linear stochastic
  approximation algorithms.
\newblock {\em IEEE Transactions on Information Theory}, 50:401--409, 2004.

\bibitem{Yurinsky95}
Y~Yurinsky.
\newblock {\em Sums and Gaussian Vectors}.
\newblock Springer-Verlag, Berlin, Heidelberg, 1995.
\newblock Lecture Notes in Mathematics, v. 1617.

\bibitem{Zhang03}
T.~Zhang.
\newblock Leave-one-out bounds for kernel methods.
\newblock {\em Neural Computation}, 15:1397--1437, 2003.

\bibitem{Zinkevich03}
M.~Zinkevich.
\newblock Online convex programming and generalized infinitesimal gradient
  ascent.
\newblock Technical report, CMU-CS-03-110, School of Computer Science, Carnegie
  Mellon University, 2003.

\end{thebibliography}

\end{document}
